%%%%%%%%%%%%%%%%%%%%%%%%%%%%%%%%%%%%%%%%%%%%%%%%%%%%%%%%%%%%%%%%%%%%
%%%%%%%%%%%%%%%%%%%%%%%%%%%%%%%%%%%%%%%%%%%%%%%%%%%%%%%%%%%%%%%%%%%%
%%  Albert Abelló Lozano MSc Thesis document                                                               %%
%%%%%%%%%%%%%%%%%%%%%%%%%%%%%%%%%%%%%%%%%%%%%%%%%%%%%%%%%%%%%%%%%%%%
%%%%%%%%%%%%%%%%%%%%%%%%%%%%%%%%%%%%%%%%%%%%%%%%%%%%%%%%%%%%%%%%%%%%

%% K‰yt‰ toinen n‰ist‰, jos kirjoitat suomeksi:
%% ensimm‰inen, jos k‰yt‰t pdflatexia (kuvat on oltava pdf-tiedostoina)
%% toinen, jos haluat tuottaa ps-tiedostoa (k‰yt‰ eps-formaattia kuville).
%%
%% Use one of these you write in Finnish:
%% the 1st when using pdflatex (use pdf figures) or
%% the 2nd when producing a ps file (use eps figures).
%\documentclass[finnish,12pt,a4paper,pdftex]{article}
%\documentclass[finnish,12pt,a4paper,dvips]{article}


%% K‰yt‰ n‰it‰, jos kirjoitat englanniksi
%%
%% Uncomment one of these if you write in English
\documentclass[english,12pt,a4paper,pdftex]{article}
%\documentclass[english,12pt,a4paper,dvips]{article}

%% T‰m‰ paketti on pakollinen
%% Valitse korkeakoulusi n‰ist‰: arts, biz, chem, elec, eng, sci.
%%
%% This package is required
%% Choose your school from arts, biz, chem, elec, eng, sci.
\usepackage[elec]{aaltothesis}

%% Jos k‰yt‰t latex-komentoa k‰‰nnett‰ess‰ (oletusarvo) 
%% kuvat kannattaa tehd‰ eps-muotoon. ƒl‰ k‰yt‰ ps-muotoisia kuvia!
%% K‰yt‰ seuraavaa latex-komennon ja eps-kuvien kanssa 
%%
%% Jos t‰‰s k‰yt‰t pdflatex-komentoa, joka k‰‰nt‰‰ tekstin suoraan
%% pdf-tiedostoksi, kuvasi on oltava jpg-formaatissa tai pdf-formaatissa.
%%
%% Use this if you run pdflatex and use jpg/pdf-format pictures.
%%
\usepackage[utf8]{inputenc} % Ficheros de entrada en latin1

\usepackage{graphicx}
\usepackage{caption}
\usepackage{subcaption}
\usepackage{subfig} 
\graphicspath{{../figures/}}
%%\DeclareGraphicsExtensions{.pdf}{png}{.png}{.jpg}{jpg}{.jpeg}{jpeg}
\DeclareGraphicsExtensions{.svg,.pdf,.png,.jpeg,.jpg}
\usepackage{url}
\usepackage{hyperref}
\usepackage[small, bf]{caption}
\usepackage{dcolumn}
\usepackage{multirow}
\usepackage[font=small]{subfig}
\usepackage{mathtools}
\usepackage{listings}
\usepackage{booktabs}
\lstdefinelanguage{JavaScript}{
  keywords={typeof, new, true, false, catch, function, return, null, catch, switch, var, if, in, while, do, else, case, break},
  keywordstyle=\color{blue}\bfseries,
  ndkeywords={class, export, boolean, throw, implements, import, this},
  ndkeywordstyle=\color{darkgray}\bfseries,
  identifierstyle=\color{black},
  sensitive=false,
  comment=[l]{//},
  morecomment=[s]{/*}{*/},
  commentstyle=\color{purple}\ttfamily,
  stringstyle=\color{red}\ttfamily,
  morestring=[b]',
  morestring=[b]"
}
\lstset{frame=tb,
  language=JavaScript,
  aboveskip=3mm,
  belowskip=3mm,
  showstringspaces=false,
  columns=flexible,
  basicstyle={\small\ttfamily},
  numbers=none,
  numberstyle=\tiny\color{gray},
  keywordstyle=\color{blue},
  commentstyle=\color{dkgreen},
  stringstyle=\color{mauve},
  breaklines=true,
  breakatwhitespace=true
  tabsize=3
}

\hypersetup{
    bookmarks=true,         % show bookmarks bar?
    unicode=false,          % non-Latin characters in Acrobat�s bookmarks
    pdftoolbar=true,        % show Acrobat�s toolbar?
    pdfmenubar=true,        % show Acrobat�s menu?
    pdffitwindow=false,     % window fit to page when opened
    pdfstartview={FitH},    % fits the width of the page to the window
    pdftitle={Performance analysis of topologies for Web-based Rea-Time Communication (WebRTC)},    % title
    pdfauthor={Albert Abello Lozano},     % author
    pdfsubject={Real-Time Media Communications},   % subject of the document
    pdfcreator={Albert Abello Lozano},   % creator of the document
    pdfnewwindow=true,      % links in new window
    colorlinks=true,       % false: boxed links; true: colored links
    linkcolor=blue,          % color of internal links (change box color with linkbordercolor)
    citecolor=black,        % color of links to bibliography
    filecolor=magenta,      % color of file links
    urlcolor=cyan           % color of external links
}
%%\usepackage{ucs} % Soporte Unicode
%% Use this if you do not like hyperref package - this
%% defines url environment and formats it correctly

%% Matematiikan fontteja, symboleja ja muotoiluja lis‰‰, n‰it‰ tarvitaan usein 
%%
%% Use this if you write hard core mathematics, these are usually needed
\usepackage{amsfonts,amssymb,amsbsy}  
\usepackage[titletoc,toc,page]{appendix}
\usepackage{nomencl}
\renewcommand{\nomname}{Abbreviatures}
\renewcommand{\nomlabel}[1]{\hspace{1em}#1}
%% Vaakasuunnan mitat, ƒLƒ KOSKE!
\setlength{\hoffset}{-1in}
\setlength{\oddsidemargin}{35mm}
\setlength{\evensidemargin}{25mm}
\setlength{\textwidth}{15cm}
%% Pystysuunnan mitat, ƒLƒ KOSKE!
\setlength{\voffset}{-1in}
\setlength{\headsep}{7mm}
\setlength{\headheight}{1em}
\setlength{\topmargin}{25mm-\headheight-\headsep}
\setlength{\textheight}{23cm}
\usepackage{xpatch}
%% Kaikki mik‰ paperille tulostuu, on t‰m‰n j‰lkeen
%%
%% Output starts here
\begin{document}

%% Korjaa vastaamaan korkeakouluasi, jos automaattisesti asetettu nimi on 
%% virheellinen 
%%
%% Change the school field to describe your school if the autimatically 
%% set name is wrong
% \university{aalto University}{aalto-Yliopisto}
% \school{School of Electrical Engineering}{S‰hkˆTekniikan korkeakoulu}

%% Vain kandityˆlle: Korjaa seuraavat vastaamaan koulutusohjelmaasi
%%
%% Only for B.Sc. thesis: Choose your degree programme. 
\degreeprogram{Electronics and electrical engineering}%
{Elektroniikka ja s‰hkˆtekniikka}
%%

%% Vain DI/M.Sc.- ja lisensiaatintyˆlle: valitse laitos, 
%% professuuri ja sen professuurikoodi. 
%%
%% Only for M.Sc. and Licentiate thesis: Choose your department,
%% professorship and professorship code. 
\department{Department of Communication and Networking}%
{Radiotieteen ja -tekniikan laitos}
\professorship{Networking Technology}{Piiriteoria}
\code{S-55}
%%

%% Valitse yksi n‰ist‰ kolmesta
%%
%% Choose one of these:
%\univdegree{BSc}
\univdegree{MSc}
%\univdegree{Lic}

%% Oma nimi
%%
%% Should be self explanatory...
\author{Albert Abell\'o Lozano}

%% Opinn‰ytteen otsikko tulee vain t‰h‰n. ƒl‰ tavuta otsikkoa ja
%% v‰lt‰ liian pitk‰‰ otsikkoteksti‰. Jos latex ryhmittelee otsikon
%% huonosti, voit joutua pakottamaan rivinvaihdon \\ kontrollimerkill‰.
%% Muista ett‰ otsikkoja ei tavuteta! 
%% Jos otsikossa on ja-sana, se ei j‰‰ rivin viimeiseksi sanaksi 
%% vaan aloittaa uuden rivin.
%% 
%% Your thesis title. If the title is very long and the latex 
%% does unsatisfactory job of breaking the lines, you will have to
%% break the lines yourself with \\ control character. 
%% Do not hyphenate titles.
\thesistitle{Performance analysis of topologies for Web-based Real-Time Communication (WebRTC)}{Opinn‰yteohje}

\place{Espoo}
%% Kandidaatintyˆn p‰iv‰m‰‰r‰ on sen esitysp‰iv‰m‰‰r‰! 
%% 
%% For B.Sc. thesis use the date when you present your thesis. 
\date{20.3.2012}

%% Kandidaattiseminaarin vastuuopettaja tai diplomityˆn valvoja.
%% Huomaa titteliss‰ "\" -merkki pisteen j‰lkeen, 
%% ennen v‰lilyˆnti‰ ja seuraavaa merkkijonoa. 
%% N‰in tehd‰‰n, koska kyseess‰ ei ole lauseen loppu, jonka j‰lkeen tulee 
%% hieman pidempi v‰li vaan halutaan tavallinen v‰li.
%%
%% B.Sc. or M.Sc. thesis supervisor 
%% Note the "\" after the comma. This forces the following space to be 
%% a normal interword space, not the space that starts a new sentence. 
\supervisor{Prof.\ J\"org Ott}{Prof.\ J\"org Ott}

%% Kandidaatintyˆn ohjaaja(t) tai diplomityˆn ohjaaja(t)
%% 
%% B.Sc. or M.Sc. thesis advisors(s). 
%%
%% Note that there has been a change in the official EN translation
%% of the Finnish title ``ohjaaja'' which in the previous version (1.5) 
%% of this document was called ``instructor''. The recommended
%% translation is now ``advisor''.  
%% However, the LaTeX internal variable remains \instructor
%% as there is little point to change the variable name. 
%%
%\instructor{Prof. Pirjo Professori}{Prof. Pirjo Professori}
\instructor{M.Sc.\ (Tech.) Varun Singh}{TkT Varun Singh}
%\instructor{M.Sc.\ (Tech.) Polli Pohjaaja}{DI Polli Pohjaaja}

%% Aaltologo: syntaksi:
%% \uselogo{aaltoRed|aaltoBlue|aaltoYellow|aaltoGray|aaltoGrayScale}{?|!|''}
%% Logon kieli on sama kuin dokumentin kieli
%%
%% Aalto logo: syntax:
% \uselogo{aaltoRed|aaltoBlue|aaltoYellow|aaltoGray|aaltoGrayScale}{?|!|''}
%% Logo language is set to be the same as the document language.
\uselogo{aaltoRed}{''}

%% Tehd‰‰n kansilehti
%%
%% Create the coverpage
\makecoverpage

%% Pakotetaan uusi sivu varmuuden vuoksi, jotta 
%% mahdollinen suomenkielinen ja englanninkielinen tiivistelm‰
%% eiv‰t tule vahingossakaan samalle sivulle
%%
%% Force new page so that English abstract starts from a new page
\newpage
%
%% English abstract, uncomment if you need one. 
%% 
%% Abstract keywords
\keywords{Resistor, Resistance,\\ Temperature}
%% Abstract text
\begin{abstractpage}[english]
 Your abstract in English. Try to keep the abstract short, approximately 
 100 words should be enough. Abstract explains your research topic, 
 the methods you have used, and the results you obtained.  
\end{abstractpage}
%% Note that 
%% if you are writting your master's thesis in English place the English
%% abstract first followed by the possible Finnish abstract

%\mysection{}
%\vspace*{\fill} 
%\begin{quote} 
%%\centering 
%%{\it ``It has become appallingly obvious that our technology has exceeded our humanity.''}
%
%
%%\hfill \textbf{Albert Einstein (1879 - 1955)}
%\end{quote}
%\vspace*{\fill}
%% Preface
\mysection{Preface}
Thank you everybody.\\

\vspace{5cm}
Otaniemi, 9.3.2012

\vspace{5mm}
{\hfill Albert Abell\'o Lozano \hspace{1cm}}

%% Pakotetaan varmuuden vuoksi esipuheen j‰lkeinen osa
%% alkamaan uudelta sivulta
%%
%% Force new page after preface
\newpage

%% Sis‰llysluettelo
%% addcontentsline tekee pdf-tiedostoon viitteen sis‰llysluetteloa varten
%% 
%% Table of contents. 
\addcontentsline{toc}{section}{Contents}


%% Tehd‰‰n sis‰llysluettelo
%%
%% Create it. 
\tableofcontents
\listoffigures
\listoftables

%%\nomenclature{$l$}{Length\nomunit{m}}
%% 
%% Corrects the page numbering, there is no need to change these
\cleardoublepage
\storeinipagenumber
%%
%%Definitions and abreviations
\setlength{\nomitemsep}{-\parsep}
\setlength{\nomlabelwidth}{.20\hsize}
\makenomenclature
\printnomenclature

%\addcontentsline{toc}{section}{Definitions and abreviations}

\newpage

\pagenumbering{arabic}
\setcounter{page}{1}

%% 
%% Leave first page empty
\thispagestyle{empty}



%%introduction chapter
\section{Introduction}

%% 
%% Leave first page empty
\thispagestyle{empty}

The need of a new way to communicate between two points of the planet is a problem that many different technologies have tried to approach. Systems such Skype or VoIP are not able to cope the needs of the new generations of developers and users that everyday require a more integrated way of communication with the World Wide Web (WWW). 

Besides this, the amount of data transferred during the last years and the prevision for the future allocates a new scenario where non-centralized systems such as P2P are required as data bandwidth grows and systems need to become more scalable. Nowadays, networks are still manly content-centric, meaning that data is provided from a source to a client in a triangle scheme, clients upload data to central servers and this data is transferred to the endpoint. This architecture has been provided since long time as reliable and scalable, but with the appearance of powerful applications and Video On Demand (VOD) scalability is becoming an issue.

Those circumstances lead to a whole new world of real-time browser based applications which require also a new framework to work with. Ranging from online videoconferencing to real-time data applications, for this purpose few attempts were made in the past being highly reliable on specific hardware and custom-built no-compatible systems. Those proposals were not accessible to normal users that could not afford to adapt the requirements. 

All previous concepts are now possible thanks to the increase of the average performance in every computer nowadays, this situation is helping to build more complex browsers that are able to perform many different tasks that enhance web browsing to a different level. Having a browser to handle OpenGL style of applications is now possible thank to the new  HyperText Markup Language version 5 (HTML5) standard. Multimedia abilities are also able to be reproduced on those browsers and webcam media shown as HTML is now a reality. Even dough, there is still an important issue that must be addressed: there is no common standardized protocol that allows developers to do this. Web Real-Time Communication (WebRTC) effort to approach this problem is to build a simple and standard solution for peer-to-peer browser communication in the HTML5 environment~\cite{alvestrandOverview2012} .

Internet bandwidth capabilities helped to take the decision to start integrating peer-to-peer solutions in browsed based applications, this is due the year-by-year increase of user bandwidth connectivity during the last 10 years. Actual latency in the network is low enough to allow real-time applications to work resiliently in the browser. The amount of users being able to transfer at high speed has increased during last years (Figure~\ref{fig:bwWorldAvg}), about 39\% of users are able to download at speeds greater than 4Mbps being this a very good average speed for multimedia content~\cite{akamaiq2}.

\begin{figure}[h]
  \centering
    \includegraphics[width=1\textwidth]{./figures/internetstats.pdf}
      \caption[Broadband over 4Mbps connectivity statistics]{Broadband connectivity statistics about the speeds over 4Mbps around the globe.}
	\label{fig:bwWorldAvg}
\end{figure}

Regarding the specs on the client side, recent surveys and statistics taken by the game manufacturer Steam prove that more than  61\% of machines are carrying 1 to 4 gigabytes of RAM and nearly 90\% of computers handle 2 to 4 core CPU with a 64 bit OS~\cite{steamStats}, this environment can easily handle media enhanced applications that require high performance for media encoding. WebRTC concept rests over multiple layers having the browser as an underlying application, a traditional browser allocates a lot of resources for running being the performance of the machine a bottleneck in some cases.

Traditionally, WebRTC concept approaches rely on the usage of plug-ins or other separate software components that make the system run smoother by avoiding one layer of processing (browser) but being non-standard and not cross-compatible, one of the most import ant concepts when designing applications nowadays. This approach has a new alternative with the arrival of the new HTML5 where WebRTC is integrated as one of the new Application Programming Interfaces (APIs) available alongside other many different interesting capabilities.

\subsection{Background}

WebRTC API is included into the HTML5, this is the fifth version of the WWW language. This version includes different API's and JavaScript codes that help the developer to easily introduce new features into their already existing WWW applications. The initial HTML version (2.0) was published in November 1995 with the only goal of delivering static content from the server to client browser~\cite{html2IETF}. HTML became de de facto format for serving web information. 

HTML is written in tag formatting to identify different elements. Those tags are then interpreted by the browser to show the different data content served by the server. During the evolution of the WWW different new features have been added to the HTML standard and new versions where published, things like JavaScript and Style Sheets increase the flexibility and features of the WWW content enhancing the final user experience.

Due to the need to extend the features of the already existing HTML4 standard, a new version was proposed in 2004 by the Mozilla Foundation and Opera Software~\cite{initialHTML5proposition}. This new proposition focused in new developing technologies that could be backwards compatible with the already existing browsers, the idea didn't make a success and was tier apart until January 2008 when the first Public Working Draft was published by the Web HyperText Application Technology Working Group (WHATWG) in the W3C~\cite{firstHTML5draft}.

This proposal had a greater reliance in modularity in order to move forward faster, this meant that some specs that were included in the initial draft moved to different working groups in the W3C. Those technologies defined in HTML5 are now in separate specifications, one of them being WebRTC. WebRTC works as an integrated API within the browser that is accessible using JavaScript and is used in conjunction with the Document Object Model (DOM) interfaces. Some of the APIs that have been developed are not part of the HTML5 W3C specification but are included into the WHATWG HTML specification.

\subsection{Challenges}

WebRTC is a real time media protocol that will be obliged to share the available resources with many other applications. Due to the short experience in WebRTC congestion situations that share the available bandwidth with other existing solutions we will find some lack of documentation or previous literature regarding this topic. 

The aim to investigate and help to develop new protocols such as WebRTC is usually backed by a lack of information that may affect some of the statements made in this thesis. Hopefully, this won't affect the development of itself and the conclusions obtained at the moment of the development of the topic. 

Considering WebRTC is being developed at the moment of writing this thesis some of the statements made in here might be different in the following versions of WebRTC meaning that the problems analyzed have been solved.

\subsection{Contribution}

Investigate how WebRTC performs in a real environment trying to evaluate the best way to set multiple peer connections able to transfer media in different network topologies. Measure the performance of WebRTC in a real environment, identifying bottlenecks related to encoding/decoding, media establishment or connection maintenance. All this should be performed in real-time over a browser by using the already existing WebRTC API.

Using metrics related to RTT, latency, packet loss and bandwidth usage we expect to understand the way WebRTC performs when handling in different environments.

\subsection{Goals}

WebRTC uses and adapts some existing technologies for real-time communication. This thesis will focus in studying how:

\begin{itemize}
	\item WebRTC performs considering different topologies using video/audio acquired form the Webcam using the API and encoded using different codec types provided by the standard.

	\item Usage of WebRTC to build a real application that can be used by final users proving that the API is ready to be deployed and is a good approach for the developer needs when building real-time applications over the web. This will be done in conjunction with other new APIs and technologies introduced with HTML5.
	
	\item Testing of different WebRTC topologies with different network constraints to observe the response of the actual existing API.
\end{itemize}

The final conclusion will cover an overall opinion and usage experience of WebRTC, providing some valuable feedback for the needs and requirements for further modifications on the API.

\subsection{Structure}

Not sure about here

%documentation and drafts
%\section{Documentation and drafts}

%% 
%% Leave first page empty
\thispagestyle{empty}


%What is webRTC
\section{Real-time Communication}

%% 
%% Leave first page empty
\thispagestyle{empty}

Real-time Communication is defined as any method of communication where users can exchange data packets (e.g media, text, etc.) with low latency in both direction. The purpose of RTC is widely seen as a way to communicate between people. This is done in a two-way scenario where both users are senders and receivers of media packets, live video is a one-way configuration with one unique source of data and one or multiple receivers. In the first RTC configuration, latency is very important in order to achieve good quality for bidirectional communication between both users whereas the live scenario can tolerate some latency in the link. In two-way communication data can be transmitted using multiple topologies, they are either peer-to-peer or using a centralized relay. 

Some other ways of transmitting data include multicast or broadcast. In the development of this thesis we do not study multicast and broadcast streaming.

\begin{figure}[h]
  \centering
    \includegraphics[width=1\textwidth]{./figures/P2P.pdf}
      \caption[Real time communication between two users over the Internet]{Real time communication between two users over the Internet.}
	\label{fig:RTC}
\end{figure}

Figure~\ref{fig:RTC} shows an RTC scenario for two users, the technology providing the communication may differ in each situation but the goal is always the same. 

RTC has a characteristic that is always common in all technologies, there must be a signaling or agreement between the two entities, either with the central node or with the other user. This procedure is used by the protocols to check the capabilities of the two entities before proceeding to send the media. The signaling channel is used to negotiate the codec agreement and NAT traversal methods that are used at the same time as all the multiple features that will be enabled in the new session, making it crucial to configure the media and data to be transmitted.

On the other hand, once signaling is done data starts to flow to the receiver, this stream may include media (audio or video) and different types of data (e.g binary, text, etc.). During the transmission we may also require some extra signaling messages to be exchanged in order to maintain the path or adapt the constraints to the present network conditions, at the same time, features might change during the session. 

%Those  messages are used for Offer/Answer model, congestion control adaptation or NAT traversal issues.

%Some RTC technologies and protocols are: telephony, mobile phone communication, radio, instant messaging (IM) \nomenclature{IM}{Instant Messaging}, Voice over IP (VoIP) \nomenclature{VoIP}{Voice over IP},  Video and Voice over IP (VVoIP) \nomenclature{VVoIP}{Video and Voice over IP}, Internet Relay Chat (IRC) \nomenclature{IRC}{Internet Relay Chat} and videoconferencing. 

%All the previous ways of communication work in real time and rely in Figure~\ref{fig:RTC} topology but with different technologies (e.g SIP, RTFMP and WebRTC). In this thesis we manly work with Internet RTC using media and data.
%Web Real-Time Communication is a technology that builds P2P applications by using a defined JavaScript API. The first announcement went public in a WG of the World Wide Web Consortium (W3C) in May 2011~\cite{webrtcW3cgroup} and started the official mailing list in April 2011~\cite{welcomeW3C}. During the first stage of discussion, the main goal was to define a public draft for the version 1 API implementation and a route timeline with the goal to publish the first version by March 2013. The first public draft of W3C came public the 27th of October 2011~\cite{originalW3Cdraft}. During this first W3C draft, only media (audio and video) could be sent over the network to other peers, it is focused in the way browsers are able to access the media devices without using any plugin or external software.
%
%Alongside to the W3C working group, the WebRTC project also joined the IETF with a WG in May 2011~\cite{webrtcIETFgroup} with the first public announcement charter done the 3th of May 2011~\cite{webrtcIETFcharter}. Milestones of the WG initially marked December 2011 as deadline to provide the information and elements required to the W3C for the API design input. On the other side, the main goals of the WG covered the definition of the communication model, session management, security, NAT traversal solution, media formats, codec agreement and data transport~\cite{webrtcIETFcharter}.
%
%One  of the most important steps during the process of standardization came the 1st of June 2011 when Google publicly released the source code of their API implementation~\cite{haraldpublicWebRTC}. 
%
%During all this period both WGs have been working alongside to provide a reliable solution to enable cross-platform applications to perform media and data P2P transfer over the browser in a plugin-free environment. The first final version of the WebRTC API is to be published at the end of 2013.
%
%Some alternatives are available to the WebRTC concept, considering the global architecture of WebRTC, Session Initiation Protocol (SIP) and Secure Real-Time Media Flow Protocol (RTMFP) are similar approaches to the same solution.

\subsection{Session Initiation Protocol (SIP)}

SIP \nomenclature{SIP}{Session Initiation Protocol} is a protocol used to create, modify and terminate multimedia sessions. This protocol features real-time communication between different peers with multiple optional extensions, those extensions allow the usage of instant messages or subscriptions to different events. SIP final Request for Comments (RFC) \nomenclature{RFC}{Request for Comments} was published in June 2004, this document describes the original functionalities and mechanisms of SIP~\cite{sipRFC}.

Other features of SIP is the ability to invite participants to already existing sessions in order to build multicast conferences. SIP also gives support for name redirection being a federated protocol regardless of the user network location.

The process of SIP includes the user location, availability, media capabilities, setup of the session and management of itself. On the other side, SIP can also be defined a suite of tools that are built together with other existing protocols such as Real-time Transport Protocol (RTP) \nomenclature{RTP}{Real-time Transport Protocol}, Real-time Transport Streaming Protocol (RTSP) \nomenclature{RTP}{Real-time Transport Streaming Protocol}, Session Description Protocol (SDP) \nomenclature{SDP}{Session Description Protocol} and Media Gateway Control Protocol (MEGACO) \nomenclature{MEGACO}{Media Gateway Control Protocol}.

SIP provides low level services to deliver messages between users, for example, it could deliver SDP messages to be negotiated between the endpoints in order to agree on the parameters of a session.

\begin{figure}[h]
  \centering
    \includegraphics[width=1\textwidth]{./figures/SIParchitecture.pdf}
      \caption[SIP session establishment example. Source~\cite{sipRFC}.]{SIP session establishment example. Source~\cite{sipRFC}.}
	\label{fig:SIParchitecture}
\end{figure}

Figure~\ref{fig:SIParchitecture} shows a typical example of a SIP scenario with a message exchange between two endpoints. SIP can use SDP Offer/Answer model for the session description or capabilities negotiation between the end-points~\cite{sdpIETF} and RTP for the media transport.

On the other side, SIP also relies on some elements called proxy servers that help to route requests to the final destination. Those proxies are represented in Figure~\ref{fig:SIParchitecture}.

SIP uses a wide range of methods that help the protocol to understand the type of message that is exchanging between peers. Figure~\ref{fig:SIParchitecture} shows an example of message exchange between endpoints. INVITE request is an indicator to the receiver that someone is trying to contact, Trying indicates that the INVITE has been received and the proxy is trying to find the correct path to the destination, Ringing message represents hold for answer from the other peer. Finally, once the receiver chooses to answer, an OK message is generated to indicate that the media session can start. 

The different media parameters are negotiated using the SDP bodies transmitted into the SIP methods previously mentioned. Those parameters are agreed between the peers in order to provide compatibility.

This protocol has been used for some time and improved due to many iterations and additions to itself. The knowledge raised from this technology helped to have a better understanding of the requirements when building real-time media protocols such as WebRTC.

\subsection{Real Time Media Flow Protocol (RTMFP) and Adobe Flash}

RTMFP  \nomenclature{RTMFP}{Real Time Media Flow Protocol} and Adobe Flash are proprietary technologies provided by Adobe, both services work together to deliver multimedia and RTC between users.

Adobe Flash is a media software that uses a plugin to work on top of the browser, it is used to build multimedia experiences for end users such as graphics, animation, games and Rich Internet Applications (RIA) \nomenclature{RIA}{Rich Internet Application}. It is widely used to stream video or audio in web applications, in order to enable this content we need to install Adobe Flash plugin on the computer. 

Adobe uses a proprietary programming language called JavaScript Flash Language (JSFL) \nomenclature{JSFL}{JavaScript Flash Language} and ActionScript. RTMFP and Adobe Flash require a plugin to work with any device, this obliges the user to install extra software that is not included in the browser, these two technologies are not standardized and are difficult to enable in some mobile devices. Flash Player is available for most platforms, except iOS devices, and is present in about 98\% of all Internet-enabled browser devices. This plugin allows developers to access media streams using external devices such as cameras and microphones to be used along with RTMFP.

This protocol is implemented by using Flash Player, Adobe Integrated Runtime (AIR) and Adobe Media Server (AMS) \nomenclature{AMS}{Adobe Media Server}~\cite{rtmfpDraft}. 

RTMFP uses Adobe Flash to provide media and data transfer between two end points over UDP~\cite{rtmfpDraft}. RTMFP requires a plugin to be installed in order to be functional being a proprietary protocol. It also handles congestion control over the path and NAT transversal issues. One of the biggest differences is that, compared to SIP, RTMFP does not provide inter-domain connectivity and both peers must be in the same working domain to be able to communicate.

Media transfer is encrypted, this feature is provided in RTMFP by using proprietary algorithms and different encryption methods. RTMFP architecture allows reconnection in case of connectivity issues and works multiplexing different media streams over the session. On the signaling part, Adobe uses an application server called Cirrus~\cite{cirrusFAQ} (Figure~\ref{fig:RTMFParchitecture}) to handle the signaling between the different participants of a session. This service provides support to handle different topologies such as: end-to-end, many-to-many and multicast. Those structures rely on the use of overlay techniques in multicast and mesh scenarios. 
 
 \begin{figure}[h]
  \centering
    \includegraphics[scale=0.4]{./figures/cirrusAdobe.pdf}
      \caption[RTMFP architecture using Cirrus]{RTMFP architecture using Cirrus.}
	\label{fig:RTMFParchitecture}
\end{figure}
 
One of the most valuable feature is the possibility to integrate P2P multicast topologies where one source sends a video to a group of receivers.

\subsection{Web Real-Time Communication (WebRTC)}

WebRTC is part of the HTML5 proposal, it is defined in the W3C~\cite{webrtcW3cgroup}~\cite{getusermediaDraft}, and enables RTC capabilities between Internet browsers using simple JavaScript APIs, providing video, audio and data P2P without the need of plugins. This API is in the process of replacing the need for installing a plugin to enable P2P communications between browsers, WebRTC uses existing standardized protocols to perform RTC. 

%The project was open sourced by Google to keep working with the IETF in order to standardize the technology~\cite{haraldpublicWebRTC}.

WebRTC provides interoperability between different browser vendors, this allow the APIs to be accessible by the developers assuring high degree of compatibility (Figure~\ref{fig:marketshare}). Some of the major browsers that include some WebRTC APIs are: Google Chrome, Mozilla Firefox and Opera. Other providers, such as Internet Explorer, are in process of building prototypes for WebRTC~\cite{IEwebRTC}. 

 \begin{figure}[h]
  \centering
    \includegraphics[width=1\textwidth]{./figures/marketshare}
      \caption[Market share of browser vendors by April 2013. Source~\cite{NetMarketShare}]{Market share of browser vendors by April 2013~\cite{NetMarketShare}.}
	\label{fig:marketshare}
\end{figure}

With WebRTC, developers can provide applications for most of the desktop devices available, mobile devices will integrate WebRTC as part of their HTML5 package to also enable RTC soon~\cite{ericssonbowser}.

WebRTC is composed of two important APIs that enable real-time features, GetUserMedia and PeerConnection. Both of them are accessible by JavaScript on the browser. 

%WebRTC APIs rely on the top of the {\it WebKit} rendering engine in Chrome and Opera, in April 2013 Google announced that is going to stop using {\it WebKit} as the rendering engine that is behind displaying web pages in Chrome. Instead, it's creating its own rendering engine named {\it Blink}~\cite{movingtoBlink}.
%It is able to solve NAT transversal environments by using a mixtures of ICE, TURN and STUN technologies. For the session description it uses a modified bundled version of SDP. The format used for packet transport is RTP and SRTP, modified WebSockets are in use for P2P DataChannel implementation to provide data transport multiplexed over the same stream. All the traffic is sent over UDP or TCP over the same port~\cite{alvestrandOverview2012}.
%
%WebRTC is part of the HTML5 package, both combined are an open cross-platform standard that aims to replace the Adobe proprietary proposal for P2P Real-Time Communication (RTC).
%
%By using HTML5 features we avoid the need of installing any extra software to be able to use real-time multimedia applications on the browser.

\subsubsection{GetUserMedia API}
\label{sec:gum}

WebRTC applications use the GetUserMedia API to allow access to media streams from local devices (video cameras and microphones). 

This API itself does not provide RTC, but provides the media to be used as simple HTML elements in any web application. 

{\it GetUserMedia} API allows developers to access local media devices using JavaScript code and generates media streams to be used either with the rest of the {\it PeerConnection} API or with the HTML5 video element for playback purposes~\cite{getusermediaDraft}. {\it GetUserMedia} is already interoperable between Google Chrome, Firefox and Opera~\cite{chromefirefoxinterop}.

{\it GetUserMedia} proposal removes the need for using Adobe Flash to access the media device and also the plugin requirement.

 \begin{figure}[h]
  \centering
    \includegraphics[scale=1]{./figures/mediastreamAPI.png}
      \caption[Media Stream object description. Source ~\cite{getusermediaDraft}]{Media Stream object description. Source~\cite{getusermediaDraft}.}
	\label{fig:mediastreamAPI}
\end{figure}

Figure~\ref{fig:mediastreamAPI} illustrates how the browser access the media and delivers the output to JavaScript. We use the {\it GetUserMedia} API to build WebRTC-enabled applications for RTC video conferencing. The video tag is an HTML5 Document Object Model (DOM) \nomenclature{DOM}{Document Object Model} element that reproduces local and remote media streams.

In Figure~\ref{fig:mediastreamAPI} {\it MediaStream} is the object returned by the {\it GetUserMedia} API methods, this object is contains {\it MediaStreamTracks} that carry the actual video and audio media. The goal of using this architecture is to be able, in the near future, to include multiple sources of video and audio multiplexed over the same stream from different devices. Different alternatives include the implementation of overlay topologies by forwarding media from one peer to the other by including different {\it MediaStreamTracks} into the same {\it MediaStream}. Furthermore, {\it MediaStreams} handle the synchronization between all the {\it MediaStreamTracks} included for proper playback at the application level, by this it assures that audio and video will be always synchronized.

{\it GetUserMedia} API works using a JavaScript fallback method, this method returns a {\it MediaStream} object to the application that is played in the HTML web application or used in the {\it PeerConnection} API. A sample example of this method can be seen in Listing~\ref{lst:listing1}.

\lstset{language=JavaScript}
\begin{lstlisting}[caption={Simple example of video and audio access using JavaScript},label={lst:listing1}]
GetUserMedia(cameraConstraints(), gotStream, function() {
	console.log("GetUserMedia failed");
});
    
function gotStream(stream) {
	//Stream is the MediaStream object returned by the API
	console.log("GetUserMedia succeeded");
  	document.getElementById("local-video").src = createObjectURL(stream);
}
\end{lstlisting}

In Listing~\ref{lst:listing1}, we are using the video and audio media from our devices to be played in an HTML video element identified as {\it local-video}. 

{\it GetUserMedia} API also allow developers to set some specific constraints for the media acquisition. This helps applications to better adapt the stream to their requirements, those {\it cameraConstraints()} are provided by a JavaScript Object Notation (JSON) \nomenclature{JSON}{JavaScript Object Notation} library.

\subsubsection{PeerConnection API}
\label{sec:pcAPI}

WebRTC uses a separate API to provide the networking support to transfer media and data to the other peers, this API is named {\it PeerConnection}~\cite{editorWebRTCdraft}. PeerConnection API bundles all the internal mechanisms of the browser that enable media and data transfer, at the same time it also handles all the exchange signaling messages with specific JavaScript methods. 

Figure~\ref{fig:webrtcExample} describes the topology used in WebRTC for a bi-directional media session, with the messages being sent either by WebSockets or by HTTP long polling. Messages are built using a modified bundled version of SDP, WebRTC signaling messages are similar to SIP as they use SDP bodies for the agreement.

%BUNDLED SDP AND SDP

SDP is widely used in SIP to provide media and NAT reversal negotiation between two different endpoints prior to establish data transmission. This protocol is used in WebRTC in a modified version that allow the usage of multiple media descriptions over a single set of Interactive Connectivity Establishment (ICE) \nomenclature{ICE}{Interactive Connectivity Establishment}. Usually, in other conditions, different media types will be described using different media descriptions.

This new feature is described as Bundle~\cite{SDPBUNDLE} and can be used along with the existing SDP Offer/Answer mechanism to negotiate the different media ("m=" lines) on the session.
%%

By using Bundled SDP, WebRTC multiplexes all the traffic using a single port, this means that media, data and monitoring messages are sent over the same port from peer to peer, traffic is sent over UDP or TCP~\cite{alvestrandOverview2012}. PeerConnection API provides signaling and NAT transversal techniques, this part is very important to guarantee a high degree of success when establishing calls in different environments.

{\it PeerConnection} P2P session establishment system works in a constrained environment designed to provide some degree of legacy for other SDP based technologies. Figure~\ref{fig:webrtcExample} shows how a WebRTC simple P2P scenario works, the server used for signaling is a web server. WebRTC scenarios do not work easily in a federated environment such as SIP.

On the other side, signaling is not standardized in WebRTC and has to be provided in the application level by the developer.

\lstset{language=JavaScript}
\begin{lstlisting}[caption=Simple example of {\it PeerConnection} using JavaScript,label={lst:listing2}]
//XXXX represents the stun server address
var pc_config = {"iceServers": [{"url": "stun:XXXX"}]};
pc = new webkitRTCPeerConnection(pc_config);
pc.onicecandidate = iceCallback1;

//Localstream is the local media obtained with the GetUserMedia API
pc.addStream(localstream);

function iceCallback1(event){
	if (event.candidate) {
		sendMessage(event.candidate);
	}
}

//When incoming candidate from the other peer we send it to the PeerConnection
pc.addIceCandidate(new RTCIceCandidate(event.candidate));

//This is fired when the remote media is received
pc.onaddstream = gotRemoteStream; 
function gotRemoteStream(e){
	document.getElementById("remote-video").src = URL.createObjectURL(e.stream);
}
\end{lstlisting}

Figure~\ref{fig:webrtcExample} does not show relay servers that provide NAT transversal solutions described in Section~\ref{sec:internals} . When developing a WebRTC application, those servers must be provided into the WebRTC {\it PeerConnection} configuration when starting a new call as seen in Listing~\ref{lst:listing2}.

 \begin{figure}[h]
  \centering
    \includegraphics[width=1\textwidth]{./figures/webrtcExample.pdf}
      \caption[WebRTC simple topology for P2P communication]{WebRTC simple topology for P2P communication.}
	\label{fig:webrtcExample}
\end{figure}

Listing~\ref{lst:listing2} represents a simple example of how to use the {\it PeerConnection} API to perform a P2P connection and start transferring media, this code works in conjunction with the code in section~\ref{sec:gum}. When building the new {\it PeerConnection} object we need to pass the JSON object {\it server} with the stun configuration for the NAT transversal process: {\it var pc\_config = \{"iceServers": [\{"url": "stun:XXXX"\}]\};}. 

\subsubsection{Control and Monitoring}
\label{sec:constraints}

Control and monitoring is an important part of all RTC protocols, this part is usually handled by the JavaScript API.

Media constraints are defined as a set of parameters that limit the media quality when processed from the devices such as microphone or webcam, those parameters are usually related to video size or frame rate in WebRTC. However, in WebRTC we can also adapt the maximum link rate through a special set of constraints.

Those parameters are implemented through the Statistics Model and Constraints defined in the W3C draft~\cite{editorWebRTCdraft}, these set of methods are part of the actual {\it PeerConnection} API defined in section~\ref{sec:pcAPI}. Once the {\it PeerConnection} is made and media is flowing we need to measure the quality of the connection, this is done by retrieving the stats provided in the Real Time Control Protocol (RTCP) \nomenclature{RTCP}{Real Time Control Protocol} messages that are being sent over the link form the remote side. We focus on those remote stats to study the status of the path and to obtain the desired metrics for monitoring~\cite{varunMetrics}. 

%The  Audio Video Profile with Feedback (AVPF) \nomenclature{AVPF}{Audio Video Profile with Feedback} provides significant improvements in the transmission of RTCP messages that are event driven rather than periodic~\cite{salvatore} like in other RTP based technologies.

To access the statistical data retrieved from the control messages we need to use the {\it getStats()} method of the {\it PeerConnection} object defined in the draft~\cite{editorWebRTCdraft}, this method allow the application to access that data in a JSON format that might require some post-processing. Statistical models are useful for the developers to monitor the status of their WebRTC applications and adjust the attributes of the {\it PeerConnection}. 

Within constraints, developers are able to change media capture configuration by setting parameters such as Frames per Second (FPS) \nomenclature{FPS}{Frames per Second} and video resolution. Other attributes can be set on the {\it PeerConnection} such as bandwidth requirements, transfer rate is automatically adjusted in WebRTC using its internal mechanisms but we can set a maximum value. 

JSON objects for camera and bandwidth constraints must be defined as in the following code.

\lstset{language=JavaScript}
\begin{lstlisting}[caption=JSON objects for constraints attributes in WebRTC]
//Media constraints in Pixels for Width and Height. Frames per Second in minFrameRate
var constraints = {
	"audio": true,
 	"video": {
  		"mandatory": {
   			"minWidth": "300",
   			"maxWidth": "640",
   			"minHeight": "200",
   			"maxHeight": "480",
   			"minFrameRate": "30"
  		},
  	"optional": []
 	}
}
//Bandwidth in kbps
var pc_constraints = {
	"mandatory": {},
 	"optional": [
 	 {
   		"bandwidth": "1000"
  	}
 	]
}
\end{lstlisting}

Both constraints objects are added to the {\it GetUserMedia} and {\it PeerConnection} methods when building the new session. Values are in pixels for the media attributes and Kbit/s for the rate configuration.

\subsubsection{Low vs High level API}

During the development of WebRTC there has been a lot of discussion in the different working groups about the API layout, those APIs have been designed using the feedback provided by the JavaScript developers.

One of the difficult parts in the standardization process has been to decide about the complexity level of the API, how much is available to be accessed by the developers and which configurations or mechanisms should be automated in the browser. After long discussion, WebRTC is now using Javascript Session Establishment Protocol (JSEP) \nomenclature{JSEP}{JavaScript Session Establishment Protocol}~\cite{jsepIETF}, this API is a low level API that gives the developers control of the signaling plane allowing each application to be used in specific environments.

% some applications give legacy to SIP or Jingle protocols meanwhile others might only work in a closed web domain. 

The media processing is done in the browser internals but most of the signaling is handled in the JavaScript plane by using JSEP methods and functions. Figure~\ref{fig:JSEP} shows the JSEP signaling model, this system extracts the signaling part leaving media transmission to the browser. However, JSEP provides mechanisms to create offers and answers, as well to apply them to a session. The way those messages are communicated to the remote side is left entirely up to the application.

Furthermore, JSEP also handles the state management of the session by building the specific SDP message that is forwarded to the other peer. NAT traversal mechanisms are activated in JSEP also, those mechanisms are described in chapter~\ref{sec:internals}.

%that provides the different candidates required by the other peer in order to build the connection, those mechanisms are defined in chapter~\ref{sec:internals}

Another interesting feature that JSEP provides is called {\it rehydration}, this process is used whenever a page that contains an existing WebRTC session is reloaded keeping the existing session alive. This technique avoid session cuts when accidentally reloading the page or with any automatic update from the web application. With {\it rehydration}, the current signaling state is stored somewhere outside the page, either on the server or in browser local storage~\cite{jsepIETF}.
 
 \begin{figure}[h]
  \centering
    \includegraphics[scale=0.9]{./figures/JSEP.pdf}
      \caption[JSEP signaling model]{JSEP signaling model.}
	\label{fig:JSEP}
\end{figure}

Low level APIs allow developers to build their own high level APIs that handle all the WebRTC protocol from media access to signaling. Those high level methods are useful to simplify the way JavaScript developers build their applications, building object oriented calls we can have JavaScript libraries that set up and maintain multiple calls at the same time. The benefits of having low level JSEP API for WebRTC is that there are the multiple possibilities to adapt WebRTC to the requirements of each specific application disregarding its advantages or disadvantages without being sensitive to pick one specific design at a time.

%We have developed our own high level API to create and handle RTC sessions and monitoring using WebRTC.

\subsubsection{Internals of WebRTC}
\label{sec:internals}

WebRTC has multiple internal mechanisms that enable the RTC in the browser level by using APIs. Those mechanisms work together to accomplish all the goals of WebRTC features, some of them are related to the network level and others to video access.

One of WebRTC main issues is NAT transversal difficulties, this problem usually affects all RTC related technologies. Interactive Connectivity Establishment (ICE) \nomenclature{ICE}{Interactive Connectivity Establishment} is a technique that helps WebRTC to decide which is the best way to bypass NATs and firewalls, ICE is widely used in media communications and has proven to be reliable when choosing the best option to enable connectivity in restrictive environments~\cite{iceIETF}.The enablers that work together with ICE for Real-time protocols are Simple Transversal Utilities for NAT (STUN) \nomenclature{STUN}{Simple Transversal Utilities for NAT} and Traversal Using Relays around NAT (TURN) \nomenclature{TURN}{Traversal Using Relays around NAT}~\cite{stunIETF}~\cite{turnIETF}.

TURN and STUN servers are usually placed outside the local network of the clients and help them to find the way to communicate with each other by discovering new open paths, the final decision is taken by the ICE mechanism. STUN server function is to discover the available IP addresses and ports that allow direct connectivity to a target machine placed behind a firewall or NAT, those interfaces are named {\it candidates}, this information is provided to the sender that processes it in order to choose the best {\it candidate}. On the other side, TURN works as a relay, this option should be always stated as the last resort when connection to no other {\it candidate} was established. TURNs work by rerouting the traffic from one peer to the other. 

All traffic in WebRTC is done over UDP and multiplexed over the same port. In case of TURN the traffic can be sent over TCP also.

Media encoding in WebRTC is done through codecs implemented inside the browser. Mandatory-to-Implement codecs for audio are G.711 and Opus. G.711 is an International Telecommunication Union (ITU) \nomenclature{ITU}{International Telecommunication Union} standard audio codec that has been used in multiple real time applications such as SIP. In real-time media applications, Opus is also a good alternative for G.711, Opus is a lossy audio compression format codec developed by the IETF and that is designed to work in real-time media applications on the Internet~\cite{opusIETF}. Opus can be easily adjusted for high and low encoding rates, applications can use additional codecs.

Along with the codecs, the audio engine for WebRTC also includes some features such as Acoustic Echo Cancellation (AEC) \nomenclature{AEC}{Acoustic Echo Cancellation} and Noise Reduction (NR) \nomenclature{NR}{Noise Reduction}. The first mechanism is a software based signal processing component that removes, in real time, the acoustic echo resulting from the voice being played out coming into the microphone (local loop), with this, WebRTC solves the issue of the audio loops with the output and input sound devices of computers. NR is a component that removes background noise associated to real time audio communications. When both mechanisms are working properly, the rate required by the audio channel is reduced as the unnecessary noise is removed from the spectrum. AEC and NR mechanisms provide a smooth audio input for WebRTC protocol.

%There has been a lot of discussion regarding video codec, two of the proposed codecs are H.264 and VP8. H.264 is a standard codec for video compression, this codec is widely used for recording and transmission of high definition video. Originally it was also selected due its high compatibility with existing devices and software, H.264 has made some controversy as it is patented and licensed by MPEG LA and may add some royalty problem for WebRTC. VP8 is a video compression codec owned by Google released and released in May 2010, VP8 is supported by Chrome, Opera and Firefox by default and is the de facto codec for WebRTC by May 2013. Later on, Google announced a VP8 patent cross-license agreement to provide royalty-free license to allow developers to implement VP8 video in their web applications~\cite{vp8Google}. This video codec is adaptive and performs well in low bandwidth links at the same time as providing royalty-free implementation.

WebRTC is not only useful for sending media, it can also provide P2P data transfer. This feature is named {\it Data Channel} and provides real time data transfer, this can be used with multiple purposes, from real time IM service to gaming, but it is interesting as {\it Data Channel} allows generic data exchange in a bidirectional way between two peers~\cite{datachanIETF}. Non-media data in WebRTC is transferred using System Control Transmission Protocol (SCTP) \nomenclature{SCTP}{System Control Transmission Protocol} encapsulated over Datagram Transport Layer Security (DTLS) \nomenclature{DTLS}{Datagram Transport Layer Security}~\cite{sctpIETF}~\cite{dtlsIETF}~\cite{datachanIETF}. 

The encapsulation of SCTP over DTLS on top of ICE/UDP provides a NAT traversal solution for data transfer that combines confidentiality, source authentication and integrity. This data transport service can operate in parallel with media transfer and is sent multiplexed over the same port. This feature of WebRTC is accessible from the JavaScript {\it PeerConnection} API by a combination of methods, functions and callbacks. 

%From the developer perspective, all the previous statements regarding security and transport are handled in the browser internals providing a simple and reliable way of sending P2P secure data over WebRTC.

WebRTC provides Secure Real-time Transport Protocol (SRTP) \nomenclature{SRTP}{Secure Real-time Transport Protocol} to allow media to be secured.The key-management for SRTP is provided by DTLS-SRTP which is an in-band keying and security parameter negotiation mechanism~\cite{salvatore}. Figure~\ref{fig:stack} illustrates the full protocol stack for WebRTC described in this chapter.

 \begin{figure}[h]
  \centering
    \includegraphics[scale=0.8]{./figures/protocolstack.pdf}
      \caption[WebRTC protocol stack. Source~\cite{salvatore}]{WebRTC protocol stack. Source~\cite{salvatore}.}
	\label{fig:stack}
\end{figure}

Quality of Service (QoS) \nomenclature{QoS}{Quality of Service} for WebRTC is also being discussed in the IETF and a draft is available with some proposals~\cite{qosWebRTCIETF}. WebRTC uses DiffServ packet marking for QoS but this is not sufficient to help prevent congestion in some environments. When using DiffServ, problems that arise are originated on the Internet Service Providers (ISPs) as they might be using their own packet marking with different DiffServ code-points, those packets are not interoperable between ISPs, there is an ongoing proposal to solve this problem by building consistent code-points~\cite{diffservIETF}. Otherwise, clients might also be sending too much data for the specified path reducing the effectiveness of DiffServ. Each specific application will mark Audio/video packets with the designed priority using DSCP mappings~\cite{qosWebRTCIETF}. 

Officially there is no congestion control mechanisms for WebRTC, the only mechanism actively used are circuit breakers for RTP~\cite{circuitbreakers}.

Furthermore, Chrome specifically uses a Google congestion control algorithm that enables congestion control mechanisms for rate adaptation~\cite{alvestrandCongestion2012}. The aim of this algorithm is to provide performance and bandwidth sharing with other ongoing conferences and applications that share the same link. This algorithm is defined in Section~\ref{sec:tests}.

%\subsection{Support}
%The following companies and organizations have supported and are actively working in the development of WebRTC standard in the W3C: Google, Mozilla and Opera~\cite{googleAnnouncement}. Other companies such as Microsoft have supported browser-to-browser solution but have published their own proposal which differs with the one published in the WebRTC WG, called CU-RTC-Web~\cite{curtcweb} which is a lower level API that claims to do everything that JSEP does.
%
%During the firsts attempts to build a reliable solution for WebRTC Ericsson Labs presented an initial API based on the preliminary work done in the WHATWG, this API was called ConnectionPeer API and required an special module to be installed in your browser~\cite{ericssonwebrtc}. Ericsson lately dropped from the effort to build it's own browser to focus in the standardization and codec discussion, leaving the API implementation to the Mozilla and Chrome teams. The original API evolved rapidly during the next months thanks to the WGs and the developer community feedback that is experimenting with the unstable API.
%
%%\subsection{Milestones}
%During the process of standardization some important moments should be remarked. In January 2012 Opera implemented the first version of WebRTC getUserMedia for accessing the camera and audio~\cite{operaannouncement}, during this year getUserMedia is available in the stable version of Opera. 
%
%Google Chrome integrated the first version of WebRTC in its DEV and Canary channels of the browser during January 2012~\cite{chromeannouncement}, in June 2012 it started moving its API to the stable channel hidden behind a flag, in November 2012 WebRTC becomes fully available in Google Chrome stable channel and is open for public usage~\cite{chromestable}. 
%
%Mozilla Firefox started working on the getUserMedia implementation early 2012 delivering the first version of media access trough API at the beginning of 2012 in the alpha version~\cite{mozillablog}, in April 2012 Mozilla published a WebRTC video demo running on Firefox in the "adler" channel~\cite{mozillawebrtc}, also supporting some primitive DataChannel API. Later in October Firefox Nightly was carrying the first unstable version of the WebRTC API including DataChannel~\cite{mozillafinal}, Mozilla announced in September 2012 that the stable version of WebRTC will be shipped along with Firefox 18 in January 2013~\cite{mozillacomming}, finally, the first public announcement of interoperability between Firefox and Chrome was done the 4th of February 2013~\cite{chromefirefoxinterop}.
%
%Some announcements done from Microsoft point out that they are also working in some implementation into Internet Explorer by using CU-RTC-Web as the default standard, at the moment only the Media API information is publicly available~\cite{microsoftcapture}.
%
%In October 2012 Ericsson announced the world's first WebRTC-enabled browser for mobile devices called "Bowser" with support for iOS and Android, this browser is able to handle WebRTC calls using RTCWeb Offer/Answer Protocol (ROAP) which is an old discontinued version of the WebRTC API that has moved to Javascript Session Establishment Protocol (JSEP). This browser also differs from the previous desktop alternatives on the codec side, it is carrying H.264 for video and G.711 for audio~\cite{ericssonbowser}. The API provided by Bowser is not fully W3C compliant.

%\subsection{Issues in WebRTC}
%
%WebRTC uses a mixture of different technologies to perform peer-to-peer communication between clients, those technologies range from SRTP, RTP, RTCP and multiple codecs that are being discussed. This scenario makes performance the key point for success in developing stable WebRTC applications. 
%
%Performance is manly related to computer capabilities and the ability to encode/decode at the same time as transferring and monitoring multiple peer connections. All those tasks are run over the browser and not directly on the OS, this is good for interoperability between platforms but bad in the performance aspect. Compared to Adobe technologies which uses a plugin, the performance they can deliver should be higher as they do not use as many application layers.
%
%Media applications are delay sensitive and require a low packet loss for its proper function, WebRTC is working on this aspect by trying to implement congestion control over the connection stablished between peers, this work is not completed yet and will arise as a problem in the near future. Packet loss due to system capacity and bandwidth are measurable in WebRTC using the Stats API, this API provides information about the PeerConnection performance and is accessible by JavaScript.
%
%Media constraints and bandwidth statistics will make a big difference in how media is acquired in WebRTC. Browsers and web applications have always tolerate some amount of delay and packet losses but this is not possible in media infrastructures for real time applications, an effort is needed to handle Quality of Service (QoS) in WebRTC to compete with RTMFP.

%\subsubsection{Quality of Service}
%
%Quality of Service (QoS) for WebRTC is being discussed and an internet draft is available with some proposals~\cite{qosWebRTCIETF}. WebRTC uses DiffServ packet marking for QoS but this is not sufficient to help prevent congestion in some environments. When using DiffServ the problem arises from the Internet Service Providers (ISPs) as they might be using their own packet marking with different DiffServ code-points, those won't be interoperability between ISPs, there is an ongoing proposal to build consistent code-points. Audio/video packets will be marked as priority using DSCP mappings with audio being more important than video or data~\cite{qosWebRTCIETF}. 
%
%The possibility to combine QoS in the transport layer with the constraints and stats of the WebRTC API will help developers to build more adaptive applications, for example, lowing the Frames per Second (FPS) in the case of high packet losses will reduce the bandwidth usage in the case of congestion of the link. This is possible thanks to the Stats API that provide the data statistics for the peer connection.
%
%Some environments will also require better QoS as their bandwidth will be lower, examples in the use case draft relate this to surveillance cameras or similar approaches~\cite{WebRTCcasesIETF}. In these cases QoS should be modified by using the API, this situation can lead also to malicious JavaScript injection that could flood the path with packets. 

\subsubsection{Security concerns}

To handle the signaling process WebRTC uses a web application server, peers exchange messages with each other through the web server in multiple different ways. By using this system WebRTC provides high flexibility for developers to allow multiple scenarios, on the other side, it also has some important security concerns~\cite{WebRTCcasesIETF}. Figure~\ref{fig:webrtcExample} presents a simple topology for a WebRTC call, the web application server handles the signaling messages to the peers and the media transport is done between them and provided by the browser.

Obviously, this system poses a range of new security and privacy challenges different from traditional VoIP systems. Considering that WebRTC APIs are able to bypass Firewalls and NAT, Denial of Services (DoS) \nomenclature{DoS}{Denial of Service} attacks can also become a threat. On the other side, malicious JavaScripts could also perform calling to unknown devices.

Browsers execute JavaScript scripts provided by the web applications, this may include malicious scripts, that in the case of WebRTC could lead to some privacy issues. In a WebRTC environment, we consider the browser to be a trusted unit and the JavaScript provided by the server to be unknown as it could execute a variety of actions in that browser. At a minimum, it should not be possible for arbitrary sites to initiate calls to arbitrary locations without user apprehension~\cite{rtcwebSecurityIETF}. To approach this issue, the user must make the decision to allow a call (and the access to its webcam media) with previous knowledge of who is requesting the access, where the media is going or both.

% \begin{figure}[h]
%  \centering
%    \includegraphics[scale=0.9]{./figures/xss.pdf}
%      \caption[Example of cross-site scripting attack]{Example of cross-site scripting attack.}
%	\label{fig:xss}
%\end{figure}

In web services, issues such as Cross-site scripting (XSS) \nomenclature{XSS}{Cross-site scripting} provide high risk of privacy vulnerability~\cite{crosssitescripting}. Those situations are given when a third-party server provides JavaScript scripts to a different domain to the one accessed. This script cannot be trusted by the original accessed domain as it could trigger browser actions that might harm privacy. For example, in WebRTC, the user could load a malicious script from a third-party entity in order to automatically build a WebRTC call to an undesired receiver without the user noticing this situation. Nowadays, browsers provide some degree of protection against XSS and do not let some scripting actions to be performed.

Other related vulnerabilities in WebRTC APIs include the possibility to establish media forwarding to a third peer, for example, once the user has accepted the access to the media, the provided JavaScript could build one {\it PeerConnection} to the receiver and an extra one to a remote peer that could store the call without the user noticing this behavior. Those problems are not only related to WebRTC and tend to happen in related protocols.

 \begin{figure}[h]
  \centering
    \includegraphics[width=1\textwidth]{./figures/idpWebRTCcall.pdf}
      \caption[WebRTC cross-domain call with Identity Provider authentication]{WebRTC cross-domain call with Identity Provider authentication.}
	\label{fig:idpWebRTCcall}
\end{figure}

Calling procedure is done using the JavaScript provided by the server, this may be a problem as the user must trust an unknown authority provider. WebRTC calling services usually rely on Hypertext Transfer Protocol Secure (HTTPS) \nomenclature{HTTPS}{Hypertext Transfer Protocol Secure} for authentication where the origin can be verified and users are verified cryptographically (DTLS-SRTP). Browser peers should be authorized before starting the media flow, this can be done by the {\it PeerConnection} itself using some Identity Provider (IdP) that supports OpenID or BrowserID to demonstrate their identity~\cite{rtcwebSecurityArchIETF}. Usually this problem is not particularly important in a closed domain, cases where both peers are in the same social network and provide their profiles to the system, those are exchanged previous to the call, but it arises as a big issue when having federated calls from different domains such in Figure~\ref{fig:idpWebRTCcall}.

If the web service is running over a trusted secure certificate and has authorized access to the media, {\it GetUserMedia} access becomes automatic after the first time under the same domain, otherwise, the user has to verify the access for each call. Once the media is acquired, the API builds the ICE candidates for media verification. Authentication and verification in WebRTC is an ongoing discussion in the working groups.

Security and privacy issues in WebRTC can be given in multiple layers of the protocol, the increment of trust for the provider gives some vulnerability issues that sometimes cannot be easily solved if the aim is to keep a flexible and open sourced real time protocol. Some use cases for WebRTC also incorporate some level of vulnerability as the JavaScript is going to be provided by a third-party, in the use case of media streaming, advertisement or call centers where service providers could pick data form the users and store them for further usage~\cite{WebRTCcasesIETF}.

\subsection{Comparison between SIP, RTMFP and WebRTC}

After describing various RTC technologies and two important alternatives for WebRTC, Table~\ref{fig:CompareRTC} is a summary of common features between SIP, RTMFP and WebRTC. In this Table~\ref{fig:CompareRTC}, common internal mechanisms are described for all of them.

\begin{table}[h]
\begin{center}
	\begin{tabular}{| l | c | c | c |}
	\hline
    	 & SIP & RTMFP & WebRTC \\ \hline
    	Plugin-enabled & No &Yes & No \\ \hline
    	Cross-domain & Yes & No & Maybe \\ \hline
    	%Android & Yes & Yes & Yes \\ \hline
    	%iOS & Yes & No & Yes \\ \hline
    	Audio & Yes & Yes & Yes \\ \hline
    	Video & Yes & Yes & Yes \\ \hline
    	Data & Yes & No & Yes \\ \hline
	NAT Traversal & Yes & Yes & Yes \\ \hline
	\hline \hline
	TURN & Yes & No & Yes \\ \hline
    	STUN & Yes & No & Yes \\ \hline
    	SDP & Yes & No & Yes \\ \hline
    	RTP & Yes & No & Yes \\ \hline
    	SRTP & Yes & No & Yes \\ \hline
    	UDP & Yes & Yes & Yes \\ \hline
    	TCP & Yes & No & Yes \\ \hline
    	SCTP & Yes & No & Yes \\ \hline
	\end{tabular}
      \caption[Features comparison between SIP, RTMFP and WebRTC]{Features comparison between SIP, RTMFP and WebRTC.}
	\label{fig:CompareRTC}
\end{center}
\end{table}

RTFMP is a proprietary protocol which means that it might have its own mechanisms other than the standardized ones stated on Table~\ref{fig:CompareRTC} to solve some of the issues.

All the protocols explained in this section are designed to provide similar real time features but in different ways, meanwhile SIP is a protocol that helped to develop some of the important technologies, such as RTP and SRTP, that are used in other technologies, is still not easily accessible for web developers. On the other side, RTMFP provides a licensed alternative for real time communication having some mechanisms not standardized and with compatibility issues between devices.

From the mobile perspective, SIP is used in mobile technology and WebRTC has announced to be compatible with future versions of iOS and Android~\cite{ericssonbowser}. Furthermore, RTMFP has active support for Android but is still not able to extend its usage to iOS platforms.

All three protocols provide NAT traversal solutions but RTMFP is the only one that provides a proprietary solution for NAT traversal that is not standardized, SIP and WebRTC use a conjunction of TURN, STUN and ICE mechanisms.
	
All of them are valid options, in this thesis we basically work with WebRTC and its related mechanisms.	

\subsection{Summary}

We can conclude that real-time media protocols have been developed over quite a long period. However, a standardized open source solution has not been provided yet.  WebRTC and SIP share a basement of principles that work together, we could state that WebRTC could not exist as it is now without the previous knowledge provided by SIP. Besides, the usability of SIP in web applications is still very complex and prohibitive for most web developers.

Furthermore, RTMFP has shown more ubiquitous availability on user's devices than any specific web browser. The main problem with RTMFP approach is that the protocol for end-to-end media path is proprietary, so interoperating with existing VoIP solutions can be inefficient and developers rely on vendor's plugins to take care of any platform incompatibilities.

WebRTC solution may not be perfect but is a good start to provide interoperable real-time solution for web applications.

%Use cases and topologies
\section{Topologies for real-time multimedia communication}
\label{sec:topologies}


%% 
%% Leave first page empty
\thispagestyle{empty}

In this chapter we discuss different possible topologies that can be used along in real time media communication.

A topology can be defined as the arrangement of the various nodes of a network together, those nodes can be connected through different links and configurations. Topologies enable different RTC architectures with optimal performance for each specific scenario. 

%In WebRTC, we want to study how they perform in the most common use cases for real time multimedia communication.

Some challenges are common in all the topologies described in this chapter. For example, NAT traversal problems decide either if the call is established or not, this problem can be solved in WebRTC with the usage of TURN and STUN, but in some restrictive environments it might be impossible to succeed with the call establishment. 

The usage of NAT traversal mechanisms in WebRTC is crucial and at the same time it increases the complexity of the browser internals. STUN and TURN servers must be reachable from the browser perspective in order to provide the ports and IP alternatives to connect, those are gathered into {\it candidates} that are evaluated by the ICE on the browser. ICE proceeds with the best option to perform the connection. 

All the previous mechanisms are supposed to provide high level of success probability but might fail in very restrictive environments. To solve some of the issues, WebRTC allows UDP and TCP (using TURN) packet transport, this is done to enable connectivity even in very restrictive environments that could have UDP packet drop mechanisms. 

For some topologies that include the establishment of multiple {\it PeerConnections} resource usage can be a big problem (e.g. mesh, one-to-many or tree). Considering that system capacity relies in how the OS architecture handles processes, CPU and memory usage of WebRTC might be seen as a constraint for those topologies. For example, in Unix based systems every tab of a browser is treated as a separate process meanwhile in other architectures this might be handled different. Media encoding usually consume most of those resources becoming a bottleneck for some scenarios.

\subsection{Point-to-Point}

The simplest topology is a permanent session between two peers, this model is widely used in telephony and provides reliable real time communication between users. In WebRTC, point-to-point topologies work only within people in the same domain opposite to many cross-domain communication alternatives such as SIP. Scenarios such as Figure~\ref{fig:SIParchitecture} are difficult to design in a WebRTC application, on the other side, Figure~\ref{fig:webrtcExample} represents the most common WebRTC point-to-point scenario.

With point-to-point topology we can have traditional dedicated paths where the resources are reserved for each call. In small Local Area Networks (LAN) \nomenclature{LAN}{Local Area Networks} we use dedicated paths between two WebRTC users, this path can go through the switch or relay but it is unlikely that is going to change the routing. For WebRTC calls over the public internet, the route can change at any time trying to use the optimal path, this is done in packet-switching technologies where the route is set up dynamically. However, the user perception is that the communication is done end-to-end without noticing any change on the network nodes. 

From usability perspective, different environments might require point-to-point topologies, direct calls between two users or real time communication for IM can be possible scenarios. 

%Use cases examples for point-to-point topologies allow communication between doctor and patient in a medical web application that is cross-platform compatible and uses an WebRTC. Communication in other cases such as citizens and authorities could also succeed in a WebRTC application.
 
\subsection{One-to-Many}

One-to-many or star topologies are one of the most common network topologies for media streaming (e.g Windows Media Server or RTMFP), this kind of topology consist on a central node that transmits streams to the rest of nodes connected to it. In the WebRTC example of Figure~\ref{fig:starExample}, the central node might be also receiving real time data in difference of the traditional streaming scenarios providing P2P communication between the peers and the central node.

 \begin{figure}[h]
  \centering
    \includegraphics[width=0.5\textwidth]{./figures/star.pdf}
      \caption[One-to-many topology for real time media]{One-to-many topology for real time media.}
	\label{fig:starExample}
\end{figure}

Star scenarios are known as a type of multicast, one source sends the media to the different clients that connect to the origin. When using this topology, the common uses are related with video and audio streaming to multiple peers, TV media and streaming conferences are popular use cases.

Live streaming is a common in-demand scenario for internet TV channels, this topology adapts to the requirements of the providers as subscribers can join the suggested channel when desired. 

Some of the problems are: high dependency of the central node and, in case of failure, the central node streaming could stop loosing all connectivity with the peers. On the other side, this topology is also good as it provides reliability in case of failure of one of the connected nodes because the rest of the network won't notice any difference on the response. 

For example, we could have a major sport even being retransmitted to the viewers by using one-to-many. Other solutions could cover the use of WebRTC to have a CEO talking to the employees with an HMTL5 web application. Music bands also could take advantage of this scenario by being able to transmit his show to the audience with feedback in real time or having the members playing from different geographic areas. All the previous examples take advantage of WebRTC by having direct feedback from the connected nodes, actual media streaming technologies do not provide this kind of communication between the viewer and the origin.

%\subsubsection{Challenges}
In star topology we have a video, audio and data streaming connection from one source to multiple devices. This might cause a huge load on the source when having multiple {\it PeerConnection} running, central node performance can be a big constraint in this scenario. Observing other topologies, in most cases, media delay on the network is not as important as other options due to the one-way communication only. In most scenarios, video and audio is not required to be received on the source, so having the media delayed a couple of seconds is not going to affect the user experience in the call. Those scenarios are one-way only use cases.

From the client perspective, the {\it PeerConnection} stablished is easy to handle as  in most cases no data is going to be sent back to the source, except the RTCP control messages.

\subsection{Many-to-Many}

Many-to-many topologies are also known as mesh, this style of topology is used in multiple VoIP systems for conferencing purposes. Conferencing systems are widely extended in enterprises for long-distance communication between employees and working groups, by this, the need of having those calls working with good response for all participants is very important.

In a full mesh topology all peers connect between them increasing the number of connections and used resources. The value of fully meshed networks rely on the number of subscribers, the amount of {\it PeerConnections} stablished in a mesh network shall be dependent on the amount of people in the conference. The number of {\it PeerConnections} can grow rapidly based on Equation~\ref{eq:meshformula}.

\begin{equation}
	\label{eq:meshformula}
	c = \frac{n(n-1)}{2}\\
	
	c \text{: Number of {\it PeerConnections}}\
		
	n \text{: Nodes in the mesh}
\end{equation}

Equation~\ref{eq:meshformula} calculates the amount of WebRTC connections required for a {\it full mesh} topology. 

\subsection{Multipoint Control Unit (MCU)}

MCU \nomenclature{MCU}{Multipoint Control Unit} is a device used to bridge streams in conferences, it multiplexes, mixes and encodes media of different sources to be sent over one gateway. MCU usage could be a good alternative when designing WebRTC infrastructures such as video conferencing, the ability to multiplex different streams into the same channel is going to directly affect on how the client performs when reproducing the video.

In real time media topologies, MCU is a common component, used as relay it helps end devices to handle less load for the sources by multiplexing all the streams of the call into the same channel, we can have multiple peers connected to the same MCU that can multiplex the media sent by all of them into one unique stream forwarded to all the participants of the call.

MCUs receive the streams from the clients and multiplex them over one unique channel, this provides good scalability from the client perspective because it is only building one connection even there are multiple peers on the conference.

Some MCUs may have to encode and decode media on the fly, this is can be difficult in real time applications but can provide different encoding options to adapt the stream output to the link conditions. Usually transcoding is not suitable for real time environments.

Drawbacks on the MCU model affect the dependency of the end nodes from the MCU, if the MCU fails to give good latency and performance, the call quality is affected and receivers do not get the expected response. Load in the MCU can be very high when multiple conferences are being stablished, this requires abundant resources and good throughput.

Point-to-point topologies do not require much resources from the service provider, but for the MCU scenario the service provider has to be able to scale properly.

\subsection{Overlay}

Overlaying media streams is the ability of a peer to forward media to a third party. Topologies that use overlay are those that require the media to be forwarded from one peer to the other, this kind of behavior is given in multiple peer topologies such as {\it hub-spoke} or {\it tree}, seen in Figure~\ref{fig:overlaytopologies}.

Generally, in multiple peer scenarios, we can combine all of the following structures to build a topology that fit our requirements.

WebRTC does not provide native support for media overlay yet, but it is planned to implement those features in future versions of the API. Traditionally overlay has also been used for media streaming over the internet.

\begin{figure}[h]
        \centering
        \begin{subfigure}[b]{0.5\textwidth}
                \centering
                \includegraphics[width=\textwidth]{./figures/hubandspoke.pdf}
                \caption{Hub-spoke topology}
                \label{fig:hubandspoke}
        \end{subfigure}%
        ~ %add desired spacing between images, e. g. ~, \quad, \qquad etc.
          %(or a blank line to force the subfigure onto a new line)
        \begin{subfigure}[b]{0.5\textwidth}
                \centering
                \includegraphics[width=\textwidth]{./figures/three.pdf}
                \caption{Tree topology}
                \label{fig:three}
        \end{subfigure}
        \caption[Overlay topologies]{Overlay topologies.}
        \label{fig:overlaytopologies}
\end{figure}

\subsubsection{Hub-spoke}

Hub-spoke distribution is a topology composed by nodes and arranged like a chariot wheel. Traffic moves along spokes that are connected to the hub at the center. This type of topology, represented in Figure~\ref{fig:hubandspoke}, is good for some scenarios as it requires less connections to perform a full mesh communication in the network. 

This is a centralized model, we might have problems if the key nodes of the topology fail. It also relies in one or multiple trunk paths that can be crucial for the success of the streaming, those paths should provide good throughput and low delay.

In some technologies that rely in hub and spoke, the central nodes are usually picked from the end users, calculating the best response from the users the system is able to select the best candidate where the rest of nodes connect to. When this happens that node is handling and forwarding more data that in a standalone call, sometimes without knowledge.

This topology uses the concept of overlay previously described. Hub-spoke environments are also used for logistics in the world, for delivering products and goods around the globe, focusing in bridges over the continents, goods in Europe are distributed within an internal network and shipped to other continents from a centralized node.  

\subsubsection{Tree}
 
Tree topology is based on a node hierarchy, the highest level of the tree consist of a single node that is connected to one or more nodes that forward the traffic to the other layers of the topology. Tree topologies are not constrained by the number of levels and can adapt to the required amount of end users as seen in Figure~\ref{fig:three}.  

This type of topologies are scalable and manageable. In case of failure it is relatively easy to identify the broken branch of the tree and repair that node.

On the other side, we can have connectivity problems if a node fails to keep the link up, all the layers under that node are going to be affected and the media forwarding will stop. Overlay is crucial for this topology that is widely used in media streaming, for real time communications, large tree topologies won't be the best candidates given the delay produced when forwarding the packets.

Topologies such as tree are not only used for media streaming but they can also be used to provide wireless coverage in difficult areas, acting as hotspots, each hop can extend the coverage of the wireless in remote areas.



%Congestion environments
\section{Performance Metrics for WebRTC}


%% 
%% Leave first page empty
\thispagestyle{empty}

This section will define the way we measure the performance in WebRTC environments, this real-time media environment will require an specific approach and some metrics to define how the protocol behaves in different topologies and scenarios. 

Different issues might affect directly how the WebRTC media performs, these range from the hardware of the clients to the state of the link. In the following chapters we will describe some of them that will be used in our study cases.

\subsection{Losses}

Loss rate indicates packet losses during the transmission or processing. Usually packet losses affect directly the performance of a call and can indicate how the link is behaving between the different peers, in our case, packet loss will be a direct indicator of the quality of the ongoing WebRTC transmission. However, the packet loss indicate that some packets are not arriving, another strong indicator that goes attached is delay as packets will arrive later prior to getting lost in the link. This indicator will show up when the link is carrying big congestion of failures. 

Some delayed packets should also be considered as losses as they won't be useful anymore for the ongoing connection, those packets won't show up in the stats as losses. In WebRTC loss rate will affect directly to the ongoing transmission as the delay range that we can tolerate is very low before the quality of the call deteriorates, some data-driven WebRTC connections will tolerate some more delay. In general case Loss Rate will be considered as a main point for recalculating the path by using faster routes. This indicator is manly attached to link quality.

Losses will be calculated in a certain period of time so we will be able to see how much loss rate we have in a certain range of time.

\begin{equation}
	\frac{PKT_{loss}(T) - PKT_{loss}(T-1)}{PKT_{received}(T) - PKT_{received}(T-1) + PKT_{loss}(T) - PKT_{loss}(T-1)}
	\label{eq:PKTloss}
\end{equation}

Equation~\ref{eq:PKTloss} calculates the estimated packet loss we might have on the link. This operation will be done every period, we will determine this period when building the testing environment.

\subsection{Round-Trip Time (RTT)}

The delay in a link can be measured form different perspectives, one-way delay indicates the time it takes for a packet to move from one peer to the other peer, this time includes different delays that are given in the link. This one-way delay is calculated form the time taken to process it in both sides (building and decoding), the lower layer delay in the client (interface and intra-layering delay), queuing delay (from the multiple buffers in the path) and propagation delay (speed of light). The sum of all those delays compose the total one-way delay.

Considering the structure of WebRTC, one of the most important delays that we will have to consider and study is the processing delay as our applications will rely in a multiple layer structure, running over the browser will affect the performance compared to other technologies that run directly over the OS. Delays in our case will be symmetric as we will be sending and receiving media, the delay will be important in order to reproduce the streams in the best quality possible and avoid decoding artifacts in the media. 

RTT will be an early indicator of congestion in a WebRTC connection, this RTT must be monitored and most important, the adequate RTT have to be defined for every connection as the clients won't be aware of the appropriate amount for good performance.

\subsection{Throughput}

Throughput will be a key metric for testing the performance of WebRTC environments, this value will show how much capacity of the link is taking each PC and stream. It is complex though as there is still no QoS implemented in WebRTC. The throughput metric is going to provide bandwidth for video/audio in each direction, we can then use this value to provide some quality metric averaging all the previous mentioned measures in order to monitor the overall quality of the call. A sudden drop of the throughput will mean that the bandwidth available for that PC has been drastically reduced, this will lead to artifacts, or in the word case, loose of communication between peers. In this specific situation ICE candidates will try to be renegotiated in order to obtain a different solution for the connection and reestablish the media with the best throughput possible.

\subsubsection{Audio streams}

When using real time media environments for bidirectional communication the user experience is a key indicator of success. One of the factors that have to be considered is the Noise Reduction (NR) and Acoustic Echo Canceler (AEC). Those mechanisms allow the call to be smooth and avoid extra noises and echoes from the speaker voice to be transmitted, in WebRTC will provide a strange behavior when measuring the throughput, when the is no speech the bytes transferred will be approximately zero, being the throughput negligible. This helps to reduce the bandwidth usage and provides a more comfortable conversation when having a call.

\subsection{Other metrics}

Besides the metrics explained in the previous sections we are keeping other important values that affect WebRTC.

CPU/RAM usages are logged in order to determine how an average system performs when running the different scenarios as some of them will be more demanding than others, this will give an approximate approach to the required resources needed.

Also call setup time and frequency of call drops will be saved, it might be important to determine an approximate call setup time since the start of the negotiation until the media arrives. By doing this, we are measuring a parameter that directly affects the user experience in WebRTC. From the other side, call drops will be counted to see the call success ratio in every scenario.

We will also calculate and pay attention to the delay variation, this is important as it affects how the user interacts with the other peer during a call. Having high delay variations led to an uncomfortable call and distortion. We will measure this variation and the amount of delay that different topologies produce.


%Simulation environment
\section{Evaluation Environment}
\label{sec:testingEnv}

%% 
%% Leave first page empty
\thispagestyle{empty}

Our evaluation environment runs tests on WebRTC. Figure~\ref{fig:evaluation_arch} describes the functional blocks used for a simple video call over WebRTC.

 \begin{figure}[h]
  \centering
    \includegraphics[width=1\textwidth]{./figures/evaluation_arch.pdf}
      \caption[Description of simple testing environment topology for WebRTC]{Description of simple testing environment topology for WebRTC.}
	\label{fig:evaluation_arch}
\end{figure}

\subsection{WebRTC client}

WebRTC clients are virtual machines that run a lightweight version of Ubuntu (Lubuntu\footnote{https://wiki.ubuntu.com/Lubuntu}) with 2GB of RAM and one CPU. This light version removes graphic acceleration providing better results in performance than compared with other distributions due to the virtualization of the graphic card. 

Clients run Chrome Dev version 27.01453.12 as a WebRTC capable browser. To avoid unexpected results due to a bug in the {\it Pulse Audio} module of Ubuntu that controls audio input in WebRTC\footnote{https://bugs.launchpad.net/ubuntu/+source/pulseaudio/+bug/1170313}, calls are done with video only, the amount of audio transferred due to the echo cancelation systems can be neglected.

\subsubsection{Connection Monitor}

Connection Monitor [{\it ConMon}] is a command line utility that relies on the transport layer and uses {\it libpcap}\footnote{http://www.tcpdump.org/} to sniff all the packets that to go a certain interface and port~\cite{singhConMon}. This utility is designed specifically to detect and capture RTP/UDP packets. {\it ConMon} detects and saves the header but discards the payload of the packet keeping the information we need for calculating our performance indicators.

Typically we run the {\it PeerConnection} between two devices and start capturing those packets using {\it ConMon} at each endpoint. The {\it PeerConnection} carries real media so the environment for testing is going to be a precise approach to a real scenario of WebRTC usage.

{\it ConMon} captures are be saved into different files allowing us to plot separate different stream rates and calculate other parameters such as delay with post processing. {\it ConMon} allows us to compare network layer and JavaScript API monitoring tools, as {\it ConMon} is working directly over the network interface and avoids all the processing that the browser internals do to send the stats to the JavaScript statistical API. Figure~\ref{fig:onetooneWifiRTCConMon} represents one video stream from the same call as Figure~\ref{fig:onetooneWifiRTC} captured from the {\it ConMon} application.

 \begin{figure}[h]
  \centering
    \includegraphics[width=1\textwidth]{./figures/onetooneWiFiConMon.pdf}
      \caption[Point-to-point WebRTC video stream throughput graph using ConMon over public WiFi at the network layer]{Point-to-point WebRTC video stream throughput graph using ConMon over public WiFi at the network layer.}
	\label{fig:onetooneWifiRTCConMon}
\end{figure}

The capture from {\it ConMon} is very accurate as it analyzes all the packets that go through an interface, this data is processed and averaged for a each period of one second. 

%This process might output some fluctuations on the graph that could distort the reality in some cases..

Furthermore, {\it ConMon} is used to provide OWD and RTT calculations for our tests, in order to do this we assure a proper synchronization between local clocks in all the peers. This is done by using the sequence number of all RTP packets captured and subtracting the timestamp stored from both sides, no RTCP data is used in this analysis.

\subsubsection{Stats API}

WebRTC statistical API provides methods to help developers access the lower layer network information at the receiver, those methods return all different types of statistics and performance indicators that we use to build high level JavaScript {\it Stats API}. When using those statistics we process all the output data to obtain the metrics for WebRTC.

This system works in parallel with {\it ConMon}, both of them can provide similar results of some metrics and the comparison might be interesting to check the differences between the browser API and an interface layer capture. By communing both methods we can verify the results and accuracy of the metrics.

The method used for {\it Stats API} is the {\it RTCStatsCallback} that returns a JSON object that has to be parsed and manipulated to get the correct indicators, this object returns as many arrays as streams available in a {\it PeerConnection}, two audio and video~\cite{editorWebRTCdraft} objects per {\it PeerConnection} when having a point-to-point call. This data is provided by the lower layers of the network channel extracting the information from the RTCP packets that come multiplexed in the same network port~\cite{rtpusageIETF}.

{\it RTCStatsCallback} is the mechanism of WebRTC that allows the developer to access different metrics, as this is still in an ongoing discussion, the stats report object has not been totally defined and can slightly change in the following versions of the WebRTC API, methods involved in the {\it RTCStatsCallback} are available on the W3C editors draft~\cite{editorWebRTCdraft}. 

We have built a high level {\it Stats API} that use those statistics from the {\it RTCStatsCallback} to calculate the RTT, throughput, loss rate and encoding/decoding rate for the different streams that are being processed. Those stats are saved into a file or sent as a JSON object to a centralized monitoring system. Our JavaScript API grabs any {\it PeerConnection} passed through the variable and starts looping a periodical iteration to collect those stats and, either plot them or save them into an array for post-processing. 

Figure~\ref{fig:onetooneWifiRTC} represents an example of a captured call between two browsers in two different machines, Mac and Ubuntu, the call was made over Wifi network with no firewall but with unknown cross traffic on it. The measures were directly obtained from the {\it Stats API} we built and post-processed using {\it gnuplot}\footnote{http://www.gnuplot.info/}.

 \begin{figure}[h]
  \centering
    \includegraphics[width=1\textwidth]{./figures/onetooneWifiStatsRTC.pdf}
      \caption[Point-to-point WebRTC video call total throughput graph using {\it Stats API} over public WiFi]{Point-to-point WebRTC video call total throughput graph using {\it Stats API} over public WiFi.}
	\label{fig:onetooneWifiRTC}
\end{figure}

Figure~\ref{fig:onetooneWifiRTC} plots the overall bandwidth of the call, this means that the input/output video and audio are measured together to check how much total bandwidth is being consumed over the duration of the call, as it is using RTCP packets to deliver the metrics to the {\it Stats API}, it takes a while to reach the average rate value until congestion mechanisms adapt the used rate to the network conditions. We can then plot all the different streams together to get an idea of how much bandwidth the {\it PeerConnection} is consuming.

% \begin{figure}[h]
%  \centering
%    \includegraphics[width=1\textwidth]{./figures/onetooneWiFIStatsVideoStreams.pdf}
%      \caption[Point-to-point WebRTC input/output video throughput graph using Stats API over public WiFi]{Point-to-point WebRTC input/output video throughput graph using Stats API over public WiFi.}
%	\label{fig:onetooneWifiRTCVideoStreams}
%\end{figure}
%
%Figure~\ref{fig:onetooneWifiRTCVideoStreams} shows the two video streams captured from the same machine, one is outgoing the local video stream meanwhile the second stream is the incoming video stream from the other peer. We have built a flexible processing system that allows us to capture and analyze all the possible combinations of streams and metrics. The timing used for the capture is provided by the TimeStamp available on the RTCP. The average bandwidth used in this scenario of point-to-point call in a standard wireless network is around 2000 Kbps per video stream. Both figures are plotted from the same original call.

% \begin{figure}[h]
%  \centering
%    \includegraphics[width=1\textwidth]{./figures/p2prttexample.pdf}
%      \caption[Point-to-point WebRTC RTT measure using Stats API over public WiFi]{Point-to-point WebRTC RTT measure using Stats API over public WiFi.}
%	\label{fig:p2prttexample}
%\end{figure}

%Our Stats API also provides extra information such as RTT and loss rate, RTT should be provided natively by the WebRTC %method but it is possible to calculate it by using the DataChannel provided by the PeerConnection, we are using this %channel to send a UNIX TimeStamp object to the other peer and take it back, when the round trip is finished we compare it %with the actual millisecond and obtain the total RTT.

%Figure~\ref{fig:p2prttexample} represents the capture of a video call between two peers, if we are able to process the JavaScript forwarding function in an optimal way this would led to a precise RTT measurement hack without the need of Stats method.

\subsubsection{Analysis of tools}

{\it Stats API} and {\it ConMon} measure the same metrics but from different layers of the operating system, this provides us some extra information in order to see how the our high level {\it Stats API} works and if it is reliable and accurate.

However, due to the periodical capture method, the output can produce strange plots as the information regarding to the next data period could be stored in the previous one when processing the averaged data on the system. This is an accuracy problem that cannot be easily solved, when looking at the graph, it is important to observe if two peaks (positive and negative) get compensated by each other, this would mean that the data has not been allocated to the correct period when plotted. This accuracy error is a problem that can be observed when comparing both {\it ConMon} and {\it Stats API} capture in Figure~\ref{fig:delay_mesh_peer1}. 

A second problem that we could face is the time it takes to the OS to process the stats form the RTCP packet and send them to the upper browser layer, at the receiver some of the stats are based on the current measurement of the metrics not in the RTP Receiver Report. Figure~\ref{fig:p2pincommingStatsConmonWifi} and~\ref{fig:p2poutgoingStatsConmonWifi} plot two video streams being captured from Stats API and {\it ConMon}.

\begin{figure}[h]
        \centering
        \begin{subfigure}[b]{0.5\textwidth}
                \centering
                \includegraphics[width=\textwidth]{./figures/p2p_incomming_cable_sample.pdf}
               \caption[Incomming stream]{Incomming stream.}
			\label{fig:p2pincommingStatsConmonWifi}
        \end{subfigure}%
        ~ %add desired spacing between images, e. g. ~, \quad, \qquad etc.
          %(or a blank line to force the subfigure onto a new line)
        \begin{subfigure}[b]{0.5\textwidth}
                \centering
                \includegraphics[width=\textwidth]{./figures/p2p_outgoing_cable_sample.pdf}
               \caption[Outgoing stream]{Outgoing stream.}
			\label{fig:p2poutgoingStatsConmonWifi}
	        \end{subfigure}
        \caption[P2P video stream comparison between {\it ConMon} and {\it Stats API}]{P2P video stream comparison between {\it ConMon} and {\it Stats API}.}
        \label{fig:p2pStatsConmon}
\end{figure}

Figure~\ref{fig:p2pStatsConmon} represents the incoming media stream from the other peer, we can see the little overhead that is not captured by the {\it Stats API} interface, as it just reads the bytes inside the payload of the packet. All the overhead is not considered when calculating the rate though {\it Stats API}. We can conclude that the real rate that WebRTC use is going to be defined by the result of {\it ConMon} instead of {\it Stats API}, but {\it Stats API} is going to be an accurate approach.

\subsection{Automated testing}

For our test scenario we have considered two options, manual and automated testing. The first test environment does not give as much accuracy due to the impossibility to iterate the test many times for the same configuration, if the second option is available the results can be averaged between all the iterations resulting in an accurate result.

In some environments, we won't be able to perform automated testing, when this happens the results won't be as accurate but they can provide a good approximation to the averaged value.

One of the main issues when building a test scenario is the media provided to the {\it GetUserMedia} input, this media must be as close to reality as possible without using a real webcam. Google Chrome provides a fake video flag that can be activated by adding {\it --use-fake-device-for-media-stream}\footnote{http://peter.sh/experiments/chromium-command-line-switches/} parameter, this video though, does not produce enough rate for our purposes.

%\begin{figure}[h]
%	\begin{minipage}{.5\textwidth}
%		\includegraphics[width=1\textwidth]{./figures/realVideoChrome.pdf}
%			\caption[Video stream bandwidth between two peers using webcam]{Video call bandwidth between two peers using webcam.}
%			\label{fig:realVideoChrome}
%	 \end{minipage}
%	 \begin{minipage}{.5\textwidth}
%		\includegraphics[width=1\textwidth]{./figures/automatedVideoChrome.pdf}
%			\caption[Video stream bandwidth between two peers using fake video]{Video stream bandwidth between two peers using fake video.}
%			\label{fig:automatedVideoChrome}
%	 \end{minipage}
%\end{figure}

 \begin{figure}[h]
  \centering
   \includegraphics[width=1\textwidth]{./figures/realVideoChrome.pdf}
     \caption[Video stream rate with SSRC 0x646227 captured using {\it Stats API} and webcam input]{Video stream rate with SSRC 0x646227 captured using {\it Stats API} and webcam input.}
	\label{fig:realVideoChrome}
%\end{figure}
 %\begin{figure}[h]
  \centering
	\includegraphics[width=1\textwidth]{./figures/automatedVideoChrome.pdf}
	\caption[Video stream rate with SSRC 0x3a4df354 captured using {\it Stats API} and Chrome default fake content]{Video stream rate with SSRC 0x3a4df354 captured using {\it Stats API} and Chrome default fake video input.}
	\label{fig:automatedVideoChrome}
\end{figure}

Figure~\ref{fig:realVideoChrome} represents the approximate bandwidth that a real video call uses when sending media to another peer, that capture shows the same stream captured from the origin and receiver {\it StatsAPI} perspective. The adequate rate rises to 2000 Kbps. On the other hand, Figure~\ref{fig:automatedVideoChrome} represents the scenario but using the built-in fake video in both clients, the rate for this case drops to an average of 250 Kbps. 

Both figures (\ref{fig:realVideoChrome} and~\ref{fig:automatedVideoChrome}) print one unique stream, identified with the Synchronization Source identifier (SSRC) \nomenclature{SSRC}{Synchronization Source identifier}, but from the sender and receiver perspective, {\it LV} identifies the source capture and {\it RV} the receiver stream rate. 

Comparing global output from Figures ~\ref{fig:realVideoChrome} and ~\ref{fig:automatedVideoChrome}, we can see that the obtained rate is very different concluding that we cannot use {\it --use-fake-device-for-media-stream} flag for our testing environment. The reason is that Google Chrome uses a bitmap system to draw the figures and components that will be rendered in the video tag and sent over the {\it PeerConnection}, this means that the amount of encoding and bandwidth used will be low compared to a real webcam as the media sent over with fake video is minimum.

To address this issue of the media streaming for our automated devices, we have built a fake input device on the peers, the procedure is described in Appendix~\ref{sec:fakeVideo}.

 \begin{figure}[h]
  \centering
    \includegraphics[width=1\textwidth]{./figures/testV4L2niklas.pdf}
      \caption[Video stream bandwidth using V4L2Loopback fake YUV file]{Video stream bandwidth using V4L2Loopback fake YUV file.}
	\label{fig:testV4L2niklas}
\end{figure}

Figure~\ref{fig:testV4L2niklas} represents the bandwidth of a fake video stream measured by our {\it Stats API} using an YUV\footnote{YUV is a color space that encodes video taking human perception into account, typically enabling transmission errors or compression artifacts to be more efficiently masked by the human perception than using RGB-representation.} video captured from a Logitech HD Pro C910 as source, resolution is 640x480 at a frame-rate of 30 fps. 

Results can be compared between Figure~\ref{fig:testV4L2niklas} and~\ref{fig:realVideoChrome}, both average rate output is approximately 2000 Kbps, which means that this procedure is a good approach to a real webcam. 

The combination of the previous fake video setup and multiple Secure Shell \nomenclature{SSH}{Secure Shell} scripts enables the automation mechanisms to run multiple tests without the need of multiple physical devices.

\subsection{TURN Server}

Our TURN server is used to pipe all the media as a relay, allowing us to apply the network constraints required for the tests to a centralized node, this machine is a Ubuntu Server 12.04 LTS with a tuned kernel adapted to perform better with {\it Dummynet}.

The TURN daemon we use is called {\it Restund}, which has been proven to be reliable for our needs, this open source STUN/TURN server works with {\it MySQL} database authentication~\cite{restund}. We have modified the source in order to have a hardcoded password making it easier for our needs.

To do so, we need to modify {\it db.c} file before compiling. Content of  method {\it restund\_get\_ha1} has to be replaced with the following line of code, where XXX is username and YYY the password we use for the TURN configuration.

\lstset{language=C}
\begin{lstlisting}[caption=Forcing a hardcoded password in our TURN server]
md5_printf(ha1, "\%s:\%s:\%s", "XXX", "myrealm", "YYY");
\end{lstlisting}

Furthermore, in order to force WebRTC to use TURN candidates we need to replace the WebRTC API server identification with our TURN machine by doing:

\lstset{language=JavaScript}
\begin{lstlisting}[caption=Configuring our TURN server in WebRTC]
var pc_config = {
	"iceServers": [{url: "turn:XXX@192.168.1.106:3478", credential:"YYY"}]
};
\end{lstlisting}

The previous object is provided to the {\it PeerConnection} object enabling the use of TURN.

The IP address points to our TURN server and the desired port (3478 by default), now all candidates are obtained through our TURN. This does not mean that the connection will run through the relay as WebRTC will try to find the best path which may override TURN, to force the usage of TURN candidates we need to drop all candidates that do not force the use of the relay.

\lstset{language=JavaScript}
\begin{lstlisting}[caption=Dropping all candidates except relay]
function onIceCandidate(event) {
	if ((event.candidate) && (event.candidate.candidate.toLowerCase().indexOf('relay')) !== -1) {
		sendMessage({
               		type: 'candidate',
               		label: event.candidate.sdpMLineIndex,
               		id: event.candidate.sdpMid,
               		candidate: event.candidate.candidate
          	 },receiver,from);
       	} else {
           	console.log("End of candidates.");
       	}
}
\end{lstlisting}

Function {\it onIceCandidate} is fired every time we get a new candidate form our STUN/TURN or WebRTC API, those candidates need to be forwarded to the other peer by using our own method {\it sendMessage} through {\it WebSockets} or similar polling methods. In this code, we are dropping all candidates except the ones containing the option {\it relay} on it, those are the candidates that force the {\it PeerConnection} to go through our TURN machine.

This part is important as it allow us to set the constraints in a middle point without affecting the WebRTC peers.

\subsubsection{Dummynet}

To evaluate the performance of WebRTC we may modify the conditions of the network path to imitate some specific environments. This is achieved using {\it Dummynet}, a command line network simulator that allow us to add bandwidth limitations, delays, packet losses and other distortions to the ongoing link~\cite{dummynetTool}.

{\it Dummynet} is an standard tool for some Linux distributions and OSX~\cite{dummynetTool}. In order to get appropriate results we need to apply the {\it Dummynet} rules in the TURN server, this machine will forward all the WebRTC traffic from one peer to the other being transparent for both ends. 

The real goal of using TURN in WebRTC is to bypass some restrictive Firewalls that could block the connection, in our case, this works as a way to centralize the traffic flow through one unique path that we can monitor and modify. From the performance perspective, when not adding any rules to the TURN, the traffic and response of WebRTC is normal without the user noticing any difference.

Some problems arise when using {\it Dummynet} in our scenario, we will use {\it VirtualBox}\footnote{VirtualBox is an x86 virtualization software package.} machines for some testing and for running TURN instance, read Appendix~\ref{sec:dummynet} for the fixes in {\it Dummynet} configuration for virtual machines.

\subsection{Application Server}

Our application server runs the Node.js instance to handle the WebRTC signaling part, this machine uses Ubuntu with a domain name specified as {\it dialogue.io}. 

This app is a common group working application that allow people to chat and video call at the same time in their own private chat rooms, we have modified it to build an specific room for our tests, this instance simply allow two users that access the page to automatically call each other and start running the JavaScript code with the built-in {\it Stats API}.

Most of this application is coded with JavaScript and uses WebSocket protocol to handle the signaling messages from peer to peer.

\subsection{Summary of tools}

Using all the previous mentioned tools together we are able to measure how WebRTC performs in a real environment, some tools have been modified according to our requirements of bandwidth and security. To process the data obtained by all those tools we use some special scripts that measure and extract the information we require form the captures, some of them are explained in Appendix~\ref{sec:scriptsWebRTC}.

%Tests
\section{Testing WebRTC}

\thispagestyle{empty}

In this chapter we will study how WebRTC performs in different use cases and topologies previously described in chapter~\ref{sec:topologies}. All tests will be done using a real working environment with the tools previously mentioned in chapter~\ref{sec:testingEnv}.

\subsection{Point-to-point}

In a point-to-point scenario we have performed different tests to calculate how the application performs. 

\subsubsection{WiFi scenario}

Firstly we have stablished a simple call between two peers that handle video and audio in an open WiFi network. This network does not carry any UDP packet filter or Firewall, the connection is performed without the need of STUN or TURN, we could easily say it is a straight forward peer-to-peer connection. The aim of this test is to observe how the captures differ between origin and receiver on the {\it StatsAPI} and {\it ConMon} layer.

 \begin{figure}[h]
  \centering
    \includegraphics[width=1\textwidth]{./figures/onetoone_wifi_statsconmon.pdf}
      \caption[Point-to-point video stream plot using StatsAPI and ConMon data over WiFi]{Point-to-point video stream plot using StatsAPI and ConMon data over WiFi.}
	\label{fig:onetooneWifistatsconmon}
\end{figure}

Figure~\ref{fig:onetooneWifistatsconmon} represents the throughput rate on the same video stream, the three lines are the comparison between local video stream in origin peer, remote video stream in receiver peer and {\it ConMon} capture of the remote video stream on the receiver peer. All three streams contain the same data but they are measured in different layers, this will help us to understand the difference of throughput that is handling the overhead of the RTP and the disruption caused by the WiFi network.

Notice that red and black colors represent the Local Video (LV) and Remote Video (RV) from the same SSRC, both captures indicate the same stream captured using {\it StatsAPI}, and the grey line plots the capture performed using {\it ConMon} of the same SSRC. It is easy to observe that both {\it StatsAPI} captures are similar, some offset is produced due to the processing time between the network layer and the browser API that returns all values. Besides this, the capture is neat and throughput at the output of the origin client and input of the receiver is similar. Capture in the network layer is more abrupt as all packets are captured and the period of calculus when plotting affects when the value is added, when having two opposite values peaks they should be balanced, meaning that the transmission in most of the period is stable and the peaks when plotting are a result of accuracy. Call duration in this test has been around five minutes. Some areas, mostly between 13.15.30 and 13.16.00, show a strange behavior of the link that might be produced by the WiFi, this throughput distortion is balanced on the WebRTC layer as the throughput delivered by the API does not change.

When we try to measure the quality of the call one important indicator is the delay, to calculate the delay we can either use the RTT measured by our {\it StatsAPI} or use the captures performed on the network layer by {\it ConMon}. The {\it ConMon} procedure will give us a high accuracy on the delay subtracting both timestamps from both of the clients, this will require to reduce the drift of the internal clock of the computers.

 \begin{figure}[h]
  \centering
    \includegraphics[width=1\textwidth]{./figures/delay_116_646227.pdf}
      \caption[Delay calculated on the same stream captured using ConMon in both ends over WiFi]{Delay calculated on the same stream captured using ConMon in both ends over WiFi.}
	\label{fig:delay_116_646227}
\end{figure}

Figure~\ref{fig:delay_116_646227} represents the delay of the stream plotted in~\ref{fig:onetooneWifistatsconmon}. We can see that the quality of the call is affected by the network distortion at the end of Figure~\ref{fig:onetooneWifistatsconmon}, this variation of the throughput delivers a high delay of more than 4 seconds during some period of time between 13:15:30 and 13:16:30, the media received at that time will not render correctly and the user experience of the call is going to be worst than at the beginning of the call. A bursty WiFi network will led to delay even the bandwidth seems to be stable.

\subsubsection{Non-constrained link test}

After seeing how WebRTC performs in WiFi  we are going to proceed with all tests in a controlled wired scenario adding different constraints to the link. This tests will be automated running ten iterations every time in order to get as much accurate results as possible.
 
 \begin{figure}[h]
  \centering
    \includegraphics[width=1\textwidth]{./figures/no_ipfw.pdf}
      \caption[Bandwidth results for non-conditioned link]{Bandwidth results for non-conditioned link.}
	\label{fig:no_ipfw}
\end{figure}

Figure~\ref{fig:no_ipfw} plots the average bandwidth of every call in a wired network without any link condition, the average bandwidth obtained in the test is 1949.7 Kbit/s with 233 Kbit/s of deviation which gives the conclusion of having approximately 1 Mbit/s standard bandwidth in a video stream for a non-conditioned link in WebRTC. Delay result in 5.1 ms with 1.5 ms deviation and RTT about 9.5 ms. Those results can be taken as standard for a non-conditioned WebRTC with high bandwidth resources. A summary of results is shown in Table~\ref{fig:p2p_no_ipfw}. Some interesting results to track is the amount of calls failed in every test, considering all those calls go through a TURN server we might be able to approximate the success rate when establishing calls. All results go along with the deviation being this an important factor, in this test without any link conditioner we might have small deviation values such as milliseconds, but when adding conditions to the link those values will grow carrying less accuracy.
Setup time is stablished as the time it take since the start of the PeerConnection object until the media stream from the other peer arrives, this value directly affects the time it takes for a user to be able to start talking, in the optimal environment it takes about 1.5 seconds to start the call. We also had zero packet losses and two calls that failed to succeed using TURN in the standard environment.

\begin{table}[h]
\begin{center}
    \begin{tabular}{c D{,}{\pm}{-1} D{,}{\pm}{-1} D{,}{\pm}{-1} }
   	 \toprule
	\textit{}
	& \multicolumn{1}{c}{\textit{Machine A}}
	& \multicolumn{1}{c}{\textit{Machine B}}
	& \multicolumn{1}{c}{\textit{Overall}}\\
	\midrule
	\textbf{CPU (\%)} & 48.76 ,2.76 & 48.83 ,2.78 & 48.79 ,2.77\\
	\textbf{Memory (\%)} & 35.98 ,0.3 & 36.43 ,0.29 & 36.21 ,0.29\\
	\textbf{Bandwidth (Kbit/s)} & 1947.61 ,232.75 & 1951.76 ,234.5 & 1949.7 ,233.62\\
	\textbf{Setup time (ms)} & 1436.33 ,25 & 1447.44 ,22.71 & 1441.88 ,24.04\\
	\textbf{RTT (ms)} & 9.49 ,2.11 & 9.64 ,2.71 & 9.57 ,2.41\\
	\textbf{Delay (ms)} & 4.84 ,1.5 & 5.4 ,1.53 & 5.12 ,1.52\\
	\bottomrule
    \end{tabular}
    \caption[P2P test with no link conditions]{P2P test with no link conditions.}
    \label{fig:p2p_no_ipfw}
\end{center}
\end{table}

Delay values in Table~\ref{fig:p2p_no_ipfw} are represented as a mean calculation of all the delay obtained in the link, thus this value is not representative of what happened in the call. Considering the example in Figure~\ref{fig:delay_116_646227} we can see that the delay can variate during the call being the mean not appropriate to measure the response against the conditions of the link. In order to observe the behavior of WebRTC in delay we have two different approaches, the mean delay with deviation and delay distribution of all calls. 

 \begin{figure}[h]
  \centering
    \includegraphics[width=1\textwidth]{./figures/total_delay_distribution_no_ipfw.pdf}
      \caption[Delay distribution in each P2P iterations with no link constraints]{Delay distribution in each P2P iterations with no link constraints.}
	\label{fig:total_delay_distribution_no_ipfw}
\end{figure}

 \begin{figure}[h]
  \centering
    \includegraphics[width=1\textwidth]{./figures/mean_deviation_delay_no_ipfw.pdf}
      \caption[Bandwidth mean and deviation for delay in each P2P iterations with no link constraints]{Mean and deviation for delay in each P2P iterations with no link constraints.}
	\label{fig:mean_deviation_delay_no_ipfw}
\end{figure}

Figure~\ref{fig:mean_deviation_delay_no_ipfw} represents the mean and deviation of delay calculated for all iterations, this delay is calculated on basis with the arrival timestamp for each packet with the captures performed in both sides by {\it ConMon}. We run an NTPD daemon to calculate the drift on the time and sync both machines.  There is small amount of drift of maximum 3ms in the worst case and as small as 1ms in the best one. In Figure~\ref{fig:total_delay_distribution_no_ipfw}, the distribution is given by the amount of packets that have some specific amount of delay, they are counted by batches of 10ms with a maximum range of 300ms. Most of the packets run with less than 25ms delay in all the iterations. The user experience with this small amount of delay with no aggressive steps in the plot will be barely negligible. Figures~\ref{fig:mean_deviation_delay_no_ipfw} and~\ref{fig:total_delay_distribution_no_ipfw} differentiate from Figure~\ref{fig:delay_116_646227} in the measurement of a global delay for an specific constraint scenario instead of just plotting a single call case, many aspects may affect the delay and the an optimal way to observe it is to plot the distribution and deviation of each iteration and try to guess a patron that repeats, Figure~\ref{fig:delay_116_646227} is good to observe just one call if we add some conditions to the link meanwhile the call is going on.

\subsubsection{Behavior in lossy environments}

We have performed some tests regarding lossy environments to see how WebRTC behaves in those, lossy situations can be given with some mobile environments with low coverage or just by having a busy link with no resources available.

We have tested the topology with 1, 5, 10 and 20\% of packet loss, according to the results in Table~\ref{fig:p2p_packet_bw} we are seeing a pretty good response from the internal algorithm up to 5\% with small effect to the bandwidth and delay. When running with 10\% loss the bandwidth drops to an average of 1140.8 Kbit/s and 162 Kbit/s deviation which is half of the corresponding amount for an standard call, this affects the quality of the link and video, 20\% loss will affect to the performance dropping the bandwidth to an average of 314.4 Kbit/s with 62 Kbit/s deviation. We can say that the video quality will be worst with lossy networks but the delay is not affected, having a delay distribution response that matches the standard case without affecting the way users will talk, quality will be worst but the call will be correct in terms of usage. All metrics are in the normal range except bandwidth.

The algorithm used in WebRTC regarding to packet loss is proven to work fine in lossy environments with the results obtained, but there is a big gap of performance in the 10\% loss network compared to the results with 20\%, it is obviously a big amount of packets but the response with 20\% is significantly better than the one with 10\%.

\begin{table}[h]
\begin{center}
    \begin{tabular}{c D{,}{\pm}{-1} D{,}{\pm}{-1} D{,}{\pm}{-1} }
   	 \toprule
	\textit{}
	& \multicolumn{1}{c}{\textit{Machine A}}
	& \multicolumn{1}{c}{\textit{Machine B}}
	& \multicolumn{1}{c}{\textit{Overall}}\\
	\midrule
	\textbf{1\% (Kbit/s)} & 1913.59 ,252.11 & 1880.24 ,261.46 & 1896.91 ,256.78\\
	\textbf{5\% (Kbit/s)} & 1609.65 ,158.46 & 1527.84 ,198.59 & 1568.74  ,178.52\\
	\textbf{10\% (Kbit/s)} & 1166.70 ,145.96 & 1114.94 ,177.88 & 1140.82 ,161.92\\
	\textbf{20\% (Kbit/s)} & 333.34 ,65.99 & 295.46 ,57.98 & 314.4 ,61.98\\
	\bottomrule
    \end{tabular}
    \caption[Averaged bandwidth with different packet loss conditions]{Averaged bandwidth with different packet loss conditions.}
    \label{fig:p2p_packet_bw}
\end{center}
\end{table}

The amount of packets lost in every test is slightly lower than the exact percentage of loss because the use of Forward Error Correction in WebRTC in Chrome, this mechanism is used to control errors in data connection with noisy channels that led to packet losses. FEC is not a must feature to implement in WebRTC but Chrome carries it as default.

When using FEC the sender encodes the message in a redundant way, by having this redundancy the receiver is able to detect a limited number of errors and autocorrect those errors without requiring retransmission.

On the other side, the sender calculates its rate based on the receive report that arrives from the receiver, if this report is not received within two times the maximum interval WebRTC congestion mechanism will consider that all packets during that period have been lost halving the rate in the sender. In order to improve response in lossy environments we could consider calculating the optimal value for this interval considering all the possible situations. Considering the congestion algorithm in WebRTC~\cite{alvestrandCongestion2012}, the rate should not vary when having between 2-10\% of packet losses. Table~\ref{fig:p2p_packet_bw} proves that this mechanism is not working properly as we are noticing reduction of rate with 5\% of packet losses, the mechanism should start modifying the rate above 10\% of packet lost calculating a new sender available bandwidth (A$_{\textrm{s}}$) using Equation~\ref{eq:RateCalc} being {\it p} the packet loss ratio.

\begin{equation}
	A_s ( i ) = A_s ( i - 1 ) \times (1 - \frac{p}{2}) 
	\label{eq:RateCalc}
\end{equation}

If the packet loss is less than 2\% the increase of bandwidth will be given by Equation~\ref{eq:RateCalc2}.

\begin{equation}
	A_s ( i ) = 1.05 \times (A_s ( i - 1 ) + 1000) 
	\label{eq:RateCalc2}
\end{equation}

\subsubsection{Delayed networks}

Another interesting situation that are given in mobile environments and queued networks is delay, we have also tested the performance of WebRTC in those conditions. We have benchmarked tests in different one-way delays, 50, 100, 200 and 500ms. In our case, the RTT results should be multiplied by two.

Delay modeling for real time applications is difficult and can be done using the timestamp of the incoming packets, the incoming frame will be delayed if the arrival time difference is larger than the timestamp difference compared to its predecessor frame.

We have noticed that the system performs badly when having even small delays up to 100ms. The response of WebRTC is to reduce the bandwidth by discarding packets, this means that the congestion control systems that act in those environments are not working correctly. On the other hand, delay output does behave correctly having a continuous delay of the according time configured in the constraints, there are no sudden increases of delay and the deviation in delay fits in the standard limits.

Table~\ref{fig:p2p_delay_bw} represents the bandwidth response to the delay conditions, it is interesting to see that the deviation with the biggest delay is smaller than expected. Only with 50ms the system will output a good quality call, when increasing delay the performance of the video will decrease. WebRTC uses VP8 codec which degrades gracefully the quality in packet loss and delay conditions but the response in this case should be better if the congestion mechanisms worked properly.

\begin{table}[h]
\begin{center}
    \begin{tabular}{c D{,}{\pm}{-1} D{,}{\pm}{-1} D{,}{\pm}{-1} }
   	 \toprule
	\textit{}
	& \multicolumn{1}{c}{\textit{Machine A}}
	& \multicolumn{1}{c}{\textit{Machine B}}
	& \multicolumn{1}{c}{\textit{Overall}}\\
	\midrule
	\textbf{50ms (Kbit/s)} & 1909.31 ,258.09 & 1917.81 ,251.62 & 1913.56 ,254.86\\
	\textbf{100ms (Kbit/s)} & 1516.07 ,263.43 & 1453.94 ,272.79 & 1485 ,268.11\\
	\textbf{200ms (Kbit/s)} & 503.71 ,116.45 & 617.92 ,142.69 & 560.82 ,129.57\\
	\textbf{500ms (Kbit/s)} & 303.58 ,59.22 & 207.77 ,32.48 & 255.67 ,45.85\\
	\bottomrule
    \end{tabular}
    \caption[Summary of averaged bandwidth with different delay conditions]{Summary of averaged bandwidth with different delay conditions.}
    \label{fig:p2p_delay_bw}
\end{center}
\end{table}

We can also observe that every iteration follows a different pattern even having an averaged result, Figure~\ref{fig:mean_deviation_bw_delay200} show the test performed at 200ms and the iterations that fail to keep a constant rate making the amount of artifacts in the video affect the quality of the call. We can certainly confirm that the methods that WebRTC should use to control the congestion in the call are not working as they should.

 \begin{figure}[h]
  \centering
    \includegraphics[width=1\textwidth]{./figures/mean_deviation_bw_delay200.pdf}
      \caption[Bandwidth mean and deviation for P2P 200 ms delay test]{Mean and deviation for P2P 200 ms delay test.}
	\label{fig:mean_deviation_bw_delay200}
\end{figure}

The problem with WebRTC relies in the usage of RTP over UDP for packet transport as UDP does not carry congestion control mechanisms that TCP does, when having real time media adapting the encoding to accommodate the varying bandwidth is difficult and cannot be done rapidly.

Low latency networks will play a big role when WebRTC extends to mobile devices and the ability to react properly to delays and packet losses will be crucial for the success of WebRTC in those environments against its competitors.

\subsubsection{Loss and delay}

Regarding P2P scenario we also tested the possibility of having a combined lossy network with delay added to it, this kind of environment could be easily found in mobile applications in low coverage areas. We have set 10\% packet loss with different delays such as 25ms, 50ms, 100ms and 200ms. In Table~\ref{fig:p2p_packet_bw} we saw an average of over 1 Mbit/s of bandwidth usage in 10\% loss environments, the result when adding delay to the constraint is  an average of barely 60 Kbit/s. Those results differ due to the difficulty of WebRTC to handle congestion in those environments. 

\begin{table}[h]
\begin{center}
    \begin{tabular}{c D{,}{\pm}{-1} D{,}{\pm}{-1} D{,}{\pm}{-1} }
   	 \toprule
	\textit{}
	& \multicolumn{1}{c}{\textit{Machine A}}
	& \multicolumn{1}{c}{\textit{Machine B}}
	& \multicolumn{1}{c}{\textit{Overall}}\\
	\midrule
	\textbf{25ms (Kbit/s)} & 72.59 ,18.54 & 70.69 ,18.09 & 71.78 ,18.32\\
	\textbf{50ms (Kbit/s)} & 59.7 ,16.84 & 60.36 ,18 & 60.03 ,17.42\\
	\textbf{100ms (Kbit/s)} & 63.3 ,19.29 & 64.82 ,20.95 & 64.06 ,20.12\\
	\textbf{200ms (Kbit/s)} & 66.89 ,20.12 & 65.66 ,19.63 & 66.27 ,19.87\\
	\bottomrule
    \end{tabular}
    \caption[Averaged bandwidth with different delay conditions with 10\% packet loss]{Averaged bandwidth with different delay conditions with 10\% packet loss.}
    \label{fig:p2p_delay_loss_bw}
\end{center}
\end{table}

Table~\ref{fig:p2p_delay_loss_bw} describes the averaged bandwidth result with not much difference in each situation.If we study the way WebRTC calculates the rate in difficult situations we can see that the sender will establish its decision on the RTT, packet loss and available bandwidth that is estimated from the receiving side using Equation~\ref{eq:RateCalc}~\cite{alvestrandCongestion2012}. Obviously the real output differs form the expected by using the formula, the reason is that even the congestion mechanism on WebRTC calculates the rate using Equation~\ref{eq:RateCalc}, the sender rate is always limited by the TCP Friendly Rate Control (TFRC) formula that is calculated using delay an packet loss ratio together~\cite{tfrc}.

 \begin{figure}[h]
  \centering
    \includegraphics[width=1\textwidth]{./figures/plr10_rtt50ms_RV.pdf}
      \caption[Remote stream bandwidth for 10\% packet loss rate and 50ms delay]{Remote stream bandwidth for 10\% packet loss rate and 50ms delay.}
	\label{fig:bw_plr10_rtt50ms}
\end{figure}

Figure~\ref{fig:bw_plr10_rtt50ms} is an example that illustrates how the rate is lowered after the beginning of the call even the bandwidth is available. This is due to the formulas and mechanisms previously described.

Carrying delay and losses in the same path will not be handled by the congestion mechanisms in WebRTC giving a low rate output for the stream.

Another interesting factor around this test is the setup time that increases to up 4.6 seconds with 200ms delay and 3 seconds with 50ms, obviously this increase will also affect mobile developers when establishing calls in delayed environments.

\subsubsection{Bandwidth and queue variations}

We have also performed a different set of tests modifying the bandwidth and queue length. For this part of the test we have chosen to run 500 Kbit/s, 1, 5 and 10 Mbit/s with different queue sizes ranging from 100 ms, 500 ms, 1s and 10 s. In total we have run 12 different tests with ten iterations each.

The queue size is set in slots to {\it Dummynet} considering each slot as standard ethernet packet of 1500 Bytes, to calculate this number we use Equation~\ref{eq:QueueSize}.

\begin{equation}
	\frac{Bandwidth (Bits)}{8 \times 1500} \times Queue (seconds)
	\label{eq:QueueSize}
\end{equation}

We have seen a good response when having big queue sizes but larger deviation in bandwidth when reducing this queue size to 100 ms or 500 ms, this produced high delays over 20 ms for every call with different distribution curves. The delay that is given to the duration of the call is not stable and will affect the media flow, this increasing curve of delay distribution is given by the small queue size which produces bursty packets to arrive to the peer having different delay conditions.

When we tested the 5 Mbit/s case we got a high delay output even the bandwidth response adapted to the constraints, delay deviation is also high and will affect the time the packets arrive with large jittering.

We will study the result of the test performed at 1 Mbit/s limitation as the maximum standard bandwidth for WebRTC is approximately 2 Mbit/s, when having 1 Mbit/s limitation WebRTC will need to adapt the actual encoding rate and bandwidth control to that amount.

Figure~\ref{fig:1mb_mean_deviation_bw} represents the bandwidth and mean plotted for all the different tests performed in the 1 Mbit/s case. We can see that the response varies in small amount of bandwidth but with large deviation, when having 500ms and 1s queue size (\ref{fig:1mb_500ms_mean_deviation_bw}) we have much more deviation in means of packets being buffered in the relay. Otherwise, when the queue size reduces to 100ms (\ref{fig:1mb_100ms_mean_deviation_bw}) the deviation gets smaller but delay response is worst.

We can compare Figure~\ref{fig:1mb_total_delay_distribution} delay distribution results for the best case (\ref{fig:1mb_10s_total_delay_distribution}) and worst case (\ref{fig:1mb_01s_total_delay_distribution}). The delay response with large queue is better due to the rapid increase of packets that carry small delay, for the 100ms queue the curve is smoother having packets ranging in all values of delay between 0 and 130ms.

Delay experience with small queue sizes will be worst in the sense of the call flow, we might experience sudden delay situations that WebRTC won't be able to handle, when having larger queue sizes we son't notice the delay variations as much as with the previous example. Having a curvy increase in delay distribution figure will result in sudden delay variations in the call. The conclusion is that WebRTC is able to adapt to low capacity networks using its codec mechanism at the same time as it should improve the congestion control systems to adapt to different buffer sizes and queuing conditions. 

The congestion mechanisms in WebRTC will stabilize the rate until the amount of delay triggers the rate change to fit the new queue state requirements. Figure~\ref{fig:1cd81aa8-bw} and~\ref{fig:1cd81aa8-delay} show the bandwidth and delay for the same stream and how the rate adapts once the queues are full increasing the delay on the packets, rate is lowered and queues get empty giving producing low delay.

 \begin{figure}[h]
  \centering
    \includegraphics[width=1\textwidth]{./figures/1cd81aa8-bw.pdf}
      \caption[Remote stream bandwidth for 1 Mbit/s and 500ms queue size]{Remote stream bandwidth for 1 Mbit/s and 500ms queue size.}
	\label{fig:1cd81aa8-bw}
\end{figure}

 \begin{figure}[h]
  \centering
    \includegraphics[width=1\textwidth]{./figures/1cd81aa8-delay.pdf}
      \caption[Stream delay for 1 Mbit/s and 500ms queue size]{Stream delay for 1 Mbit/s and 500ms queue size.}
	\label{fig:1cd81aa8-delay}
\end{figure}

Studying the way the sender takes decisions about rate constraints we can observe that the available bandwidth estimates calculated by the receiving side are only reliable when the size of the queues along the channel are large enough~\cite{alvestrandCongestion2012} . When having short queues along the path the maximum usage of the bandwidth cannot be estimated if there is no packet loss in the link, as in this case the packet loss is negligible, the connection is not able to use the maximum amount of bandwidth available in the link.

\begin{figure}[htp]
        \centering
        \begin{subfigure}[b]{1\textwidth}
                \centering
                \includegraphics[width=\textwidth]{./figures/1mb_10s_mean_deviation_bw.pdf}
      \caption[1 Mbit/s and 10s queue size]{1 Mbit/s and 10s queue size.}
	\label{fig:1mb_10s_mean_deviation_bw}
        \end{subfigure}
        \qquad

        \begin{subfigure}[b]{1\textwidth}
                \centering
                \includegraphics[width=\textwidth]{./figures/1mb_1s_mean_deviation_bw.pdf}
      \caption[1 Mbit/s and 1s queue size]{1 Mbit/s and 1s queue size.}
	\label{fig:1mb_1s_mean_deviation_bw}
        \end{subfigure}
        
        \begin{subfigure}[b]{1\textwidth}
                \centering
                \includegraphics[width=\textwidth]{./figures/1mb_05s_mean_deviation_bw.pdf}
      \caption[1 Mbit/s and 500ms queue size]{1 Mbit/s and 500ms queue size.}
	\label{fig:1mb_500ms_mean_deviation_bw}
        \end{subfigure}
        \qquad

        \begin{subfigure}[b]{1\textwidth}
                \centering
                \includegraphics[width=\textwidth]{./figures/1mb_01s_mean_deviation_bw.pdf}
      \caption[1 Mbit/s and 100ms queue size]{1 Mbit/s and 100ms queue size.}
	\label{fig:1mb_01s_mean_deviation_bw}
        \end{subfigure}
        \caption{Bandwidth and mean for 1 Mbit/s with multiple queue sizes}
        \label{fig:1mb_mean_deviation_bw}
\end{figure}

\begin{figure}[htp]
        \centering
        \begin{subfigure}[h]{1\textwidth}
                \centering
                \includegraphics[width=\textwidth]{./figures/1mb_10s_total_delay_distribution.pdf}
      \caption[1 Mbit/s and 10s queue size]{1 Mbit/s and 10s queue size.}
	\label{fig:1mb_10s_total_delay_distribution}
        \end{subfigure}
        
        \begin{subfigure}[h]{1\textwidth}
                \centering
                \includegraphics[width=\textwidth]{./figures/1mb_1s_total_delay_distribution.pdf}
      \caption[1 Mbit/s and 1s queue size]{1 Mbit/s and 1s queue size.}
	\label{fig:1mb_1s_total_delay_distribution}
        \end{subfigure}

        \begin{subfigure}[h]{1\textwidth}
                \centering
                \includegraphics[width=\textwidth]{./figures/1mb_05s_total_delay_distribution.pdf}
      \caption[1 Mbit/s and 500ms queue size]{1 Mbit/s and 500ms queue size.}
	\label{fig:1mb_05s_total_delay_distribution}
        \end{subfigure}%
        
        \begin{subfigure}[h]{1\textwidth}
                \centering
                \includegraphics[width=\textwidth]{./figures/1mb_01s_total_delay_distribution.pdf}
      \caption[1 Mbit/s and 100ms queue size]{1 Mbit/s and 100ms queue size.}
	\label{fig:1mb_01s_total_delay_distribution}
        \end{subfigure}
        \caption{Delay distribution for 1 Mbit/s with multiple queue sizes}
        \label{fig:1mb_total_delay_distribution}
\end{figure}

\clearpage
\clearpage
\subsection{Loaded network}

Similar to the previous test, in this case we will be measuring the performance of WebRTC in a loaded network using a tool named {\it Iperf}. This tool will allow us to emulate traffic between our two peers loading the network according to our needs with UDP or TCP packets. The configuration we will use is the one shown in Figure~\ref{fig:iperfTest} with the clients running {\it Dummynet} instead of the relay. This scenario is chosen due its widely usage in real devices, having video calls meanwhile manipulating large amounts of online data is something that might happen when using WebRTC.

 \begin{figure}[h]
  \centering
    \includegraphics[width=0.8\textwidth]{./figures/IPERF.pdf}
      \caption[Topology for traffic flooded path using {\it Iperf}]{Topology for traffic flooded path using {\it Iperf}.}
	\label{fig:iperfTest}
\end{figure}

In this scenario we are interested in measuring also the behavior of real bandwidth setups for different environments, we will be testing the link with 100/10 Mbit/s and 20/4 Mbit/s limitations, the second one could be defined as the standard for HSPA networks. The data that will be sent to the other peer will be either 10 Mbit/s of TCP and UDP traffic or 2 Mbit/s.

First we will run the server as daemon on the recipient of the packets by executing:

\begin{verbatim}
# iperf -s -D
\end{verbatim}

The next step will rely on the usage of UDP or TCP, {\it Iperf} sends TCP packets by default, to do so we will run:

\begin{verbatim}
# iperf -c XXXX -t 300 {-u} -b 10m/2m 
\end{verbatim}

In the previous command, {\it -t} is the amount of time the test length, {\it -c} is the feature that configures the remote server to send the packets to, {\it -u} is going to be used to sent UDP datagrams instead of TCP and {\it -b} will define the amount of Mbit/s to be sent to the remote server. In this case every test is run three times.

Table~\ref{fig:tcp_iperf_no_dummynet} summarizes the results of the 10 Mbit/s TCP packet test without {\it Dummynet} constraints in the link.

\begin{table}[h]
\begin{center}
    \begin{tabular}{c D{,}{\pm}{-1} D{,}{\pm}{-1} D{,}{\pm}{-1} }
   	 \toprule
	\textit{}
	& \multicolumn{1}{c}{\textit{Machine A}}
	& \multicolumn{1}{c}{\textit{Machine B}}
	& \multicolumn{1}{c}{\textit{Overall}}\\
	\midrule
	\textbf{CPU (\%)} & 81.06 ,5.2 & 82.15 ,5.23 & 81.16 ,5.22\\
	\textbf{Memory (\%)} & 35.65 ,0.43 & 34.27 ,0.39 & 34.96 ,0.41\\
	\textbf{Bandwidth (Kbit/s)} & 990.11 ,202.62 & 1250.13 ,264.38 & 1120.12 ,233.506\\
	\textbf{Setup time (ms)} & 1533.66 ,11.3 & 1577.66 ,41.89 & 1555 ,32.59\\
	\textbf{RTT (ms)} & 25.61 ,16.02 & 24.76 ,14.11 & 25.19 ,15.07\\
	\textbf{Delay (ms)} & 81.61 ,11.42 & 83.99 ,11.42 & 81.61 ,11.42\\
	\bottomrule
    \end{tabular}
    \caption[IPERF 10 Mbit/s TCP test without link constraints]{IPERF 10 Mbit/s TCP test without link constraints.}
    \label{fig:tcp_iperf_no_dummynet}
\end{center}
\end{table}

The bandwidth rate in the call is affected by the traffic of TCP packets along the path, at the same time we are getting higher delays. Call behavior in this environment changes in every iteration being unpredictable, Figure~\ref{fig:iperfTestStd} represents the bandwidth mean and deviation of every iteration, we can easily observe that in the worst case we are getting three times less rate than the optimum case, ranging from 1.5 Mbit/s to under 400 Kbit/s in the worst iteration.

 \begin{figure}[h]
  \centering
    \includegraphics[width=1\textwidth]{./figures/iperf_std_mean_deviation_bw.pdf}
      \caption[Bandwidth mean and deviation for 10 Mbit/s TCP {\it Iperf} test without link constraints]{Bandwidth mean and deviation for 10 Mbit/s TCP {\it Iperf} test without link constraints.}
	\label{fig:iperfTestStd}
\end{figure}

We can observe some interesting behavior in all three iterations when looking at Figure~\ref{fig:iperfTestStdDelay} total delay distribution, the response varies from all three tests being all of them bad, a lot of sudden delay changes will appear during the call making real time communication difficult. The delay deviation is small but the tolerance for TCP flooded networks is low int WebRTC.

 \begin{figure}[h]
  \centering
    \includegraphics[width=1\textwidth]{./figures/iperf_std_total_delay_distribution.pdf}
      \caption[Total delay distribution for 10 Mbit/s TCP {\it Iperf} test without link constraints]{Total delay distribution for 10 Mbit/s TCP {\it Iperf} test without link constraints.}
	\label{fig:iperfTestStdDelay}
\end{figure}

Now we will test the behavior when sending those 10 Mbit/s with UDP and TCP in a constrained link of 100/10 (downlink/uplink), in this test {\it Dummynet} scripts have been executed on the client side instead of in the Relay. Table~\ref{fig:tcp_iperf_100in_10out} shows the different bandwidth responses between TCP and UDP traffic, in both cases the link constraint have been the same but the result varies. We will se an increase of rate with TCP flooded packets but also an increase of delay, this delay might be produced due the need of processing more packets with TCP than the simple mechanism of UDP.

\begin{table}[h]
\begin{center}
    \begin{tabular}{c D{,}{\pm}{-1} D{,}{\pm}{-1} D{,}{\pm}{-1} }
   	 \toprule
	\textit{}
	& \multicolumn{1}{c}{\textit{Machine A}}
	& \multicolumn{1}{c}{\textit{Machine B}}
	& \multicolumn{1}{c}{\textit{Overall}}\\
	\midrule
	\textbf{Bandwidth UDP (Kbit/s)} & 159.41 ,28.69 & 149.04 ,25.76 & 159.23 ,27.23 \\
	\textbf{Delay UDP (ms)} & 98.07 ,3.14 & 98.85 ,2.75 & 96.85 ,2.94 \\
	\hline
	\hline
	\textbf{Bandwidth TCP (Kbit/s)} & 208.97 ,20.64 & 194.41 ,18.9 & 201.69 ,19.77\\
	\textbf{Delay TCP (ms)} & 146.9 ,4.38 & 147.92 ,4 & 147.41 ,4.23 \\
	\bottomrule
    \end{tabular}
    \caption[IPERF 10 Mbit/s TCP and UDP test with constrained 100/10 Mbit/s link]{IPERF 10 Mbit/s TCP and UDP test with constrained 100/10 Mbit/s link.}
    \label{fig:tcp_iperf_100in_10out}
\end{center}
\end{table}

Delay distribution response in a constrained environment (Figure~\ref{fig:10m_udp_total_delay_distribution} and \ref{fig:10m_tcp_total_delay_distribution}) is smoother compared to Figure~\ref{fig:iperfTestStdDelay}, the absolute amount of delay is larger but the distribution curve is better for WebRTC needs as it does not have any sudden increase of delay. Delay response when having constraints will output a better delay distribution but with higher RTT in the link.

\begin{figure}
        \centering
        \begin{subfigure}[b]{0.5\textwidth}
                \centering
                \includegraphics[width=\textwidth]{./figures/10m_udp_total_delay_distribution.pdf}
                \caption{Delay distribution response for UDP test}
                \label{fig:10m_udp_total_delay_distribution}
        \end{subfigure}%
        ~ %add desired spacing between images, e. g. ~, \quad, \qquad etc.
          %(or a blank line to force the subfigure onto a new line)
        \begin{subfigure}[b]{0.5\textwidth}
                \centering
                \includegraphics[width=\textwidth]{./figures/10m_tcp_total_delay_distribution.pdf}
                \caption{Delay distribution response for TCP test}
                \label{fig:10m_tcp_total_delay_distribution}
        \end{subfigure}
        \caption[10 Mbit/s UDP/TCP {\it Iperf} test with 100/10 link condition]{10 Mbit/s UDP/TCP {\it Iperf} test with 100/10 link condition.}
        \label{fig:10m_tcp_udp_distribution}
\end{figure}

When testing the 2 Mbit/s TCP and UDP flows with 20/4 Mbit/s constraints results are surprisingly close to the version without constraints, we are testing this configuration due to its similitudes to HSDPA networks that carry a similar averaged bandwidth. Unstable bandwidth is also noticed in this test but values for the rate are much higher and delay distribution graphs are similar to Figure~\ref{fig:10m_tcp_udp_distribution}. We are using 2 Mbit/s flows to imitate the encoding rate for an online streaming 1280x720 HD video.\footnote{http://www.adobe.com/devnet/adobe-media-server/articles/dynstream_live/popup.html}

Table~\ref{fig:tcp_iperf_20in_4out} describes the output we had in terms of rate and delay for the 2 Mbit/s test in a HSDPA type network. Rate adaptation is good even having an small uplink capacity of 4 Mbit/s, the way the rate is adapted to this link confirms that in this kind of not delayed or lossy low latency networks WebRTC could perform properly with simultaneous ongoing traffic. 

\begin{table}[h]
\begin{center}
    \begin{tabular}{c D{,}{\pm}{-1} D{,}{\pm}{-1} D{,}{\pm}{-1} }
   	 \toprule
	\textit{}
	& \multicolumn{1}{c}{\textit{Machine A}}
	& \multicolumn{1}{c}{\textit{Machine B}}
	& \multicolumn{1}{c}{\textit{Overall}}\\
	\midrule
	\textbf{Bandwidth UDP (Kbit/s)} & 683.81 ,259.38 & 749.66 ,249.69 & 716.74 ,254.53 \\
	\textbf{Delay UDP (ms)} & 56.34 ,2.83 & 54.31 ,2.64 & 55.32 ,2.74 \\
	\hline
	\hline
	\textbf{Bandwidth TCP (Kbit/s)} & 760.94 ,238.44 & 1174.95 ,235.12 & 967.94 ,236.78\\
	\textbf{Delay TCP (ms)} & 85.18 ,2.3 & 80.04 ,2.26 & 82.61 ,2.28 \\
	\bottomrule
    \end{tabular}
    \caption[IPERF 2 Mbit/s TCP and UDP test with constrained 20/4 Mbit/s link]{IPERF 2 Mbit/s TCP and UDP test with constrained 20/4 Mbit/s link.}
    \label{fig:tcp_iperf_20in_4out}
\end{center}
\end{table}

From the delay distribution point of view (Figure~\ref{fig:2m_tcp_udp_distribution}), the output is similar in both tests being TCP slightly better (\ref{fig:2m_tcp_total_delay_distribution}) with less absolute delay and with an acceptable variation.

\begin{figure}[h]
        \centering
        \begin{subfigure}[b]{0.5\textwidth}
                \centering
                \includegraphics[width=\textwidth]{./figures/2m_udp_total_delay_distribution.pdf}
                \caption{Delay distribution response for UDP test}
                \label{fig:2m_udp_total_delay_distribution}
        \end{subfigure}%
        ~ %add desired spacing between images, e. g. ~, \quad, \qquad etc.
          %(or a blank line to force the subfigure onto a new line)
        \begin{subfigure}[b]{0.5\textwidth}
                \centering
                \includegraphics[width=\textwidth]{./figures/2m_tcp_total_delay_distribution.pdf}
                \caption{Delay distribution response for TCP test}
                \label{fig:2m_tcp_total_delay_distribution}
        \end{subfigure}
        \caption[2 Mbit/s UDP and TCP {\it Iperf} test with 20/4 link condition]{2 Mbit/s UDP and TCP {\it Iperf} test with 20/4 link condition.}
        \label{fig:2m_tcp_udp_distribution}
\end{figure}

In general, the response of WebRTC congestion mechanisms with ongoing link traffic should be better as this environment will be common for all users. The bandwidth mechanism produces an acceptable call rate but should produce delays smaller than one second which are acceptable from the usability perspective, the delay distribution for the standard case with an ongoing traffic of 10 Mbit/s is not as good as expected but it might be due to the high capacity on the path and the way {\it Iperf} simulates the traffic.

\subsection{Parallel calls}

In this part of the test we will be checking how WebRTC handles multiple parallel calls with different peers, this is not to me mixed with mesh style of topology as it will be running using different tabs or processes through the same TURN path. Figure~\ref{fig:parallelCalls} represents the topology used for the test.

 \begin{figure}[h]
  \centering
    \includegraphics[width=1\textwidth]{./figures/ParallelCalls.pdf}
      \caption[Topology for three different parallel calls using the same link]{Topology for three different parallel calls using the same link.}
	\label{fig:parallelCalls}
\end{figure}

We will run a combined batch of tests using 2 and 3 simultaneous calls without {\it Dummynet} or with 20 Mbit/s and 10 Mbit/s bandwidth limitation for the link. The case without any constraint will run with the standard 100 Mbit/s of the ethernet link capacity. For the test we have focused in running the calls in the same machine but in different processes.

This kind of environment will be given in local networks or it could be compared with mesh topologies handling multiple peer connections, from the resources perspective it will be interesting to observe the CPU and memory consumption as every PeerConnection will be working in a different process, the machine used carries 1 CPU and 2 Gb of RAM.

\begin{table}[h]
\begin{center}
    \begin{tabular}{c D{,}{\pm}{-1} D{,}{\pm}{-1} D{,}{\pm}{-1} }
   	 \toprule
	\textit{}
	& \multicolumn{1}{c}{\textit{CPU (\%)}}
	& \multicolumn{1}{c}{\textit{Memory (\%)}}\\
	\midrule
	\textbf{Three calls} & 99.25 ,2.41 & 44.99 ,0.5 \\
	\textbf{Two calls 20 Mbit/s} & 95.67 ,3.51 & 46.16 ,0.37 \\
	\textbf{Two calls 10 Mbit/s} & 86.83 ,5.03 & 44.91 ,0.32 \\
	\textbf{Two calls} & 81.6 ,6.48 & 42.61 ,0.35 \\
	\bottomrule
    \end{tabular}
    \caption[Memory and CPU consumption rates for parallel calls in different link conditions]{Memory and CPU consumption rates for parallel calls in different link conditions.}
    \label{fig:cpu_mem_parallel}
\end{center}
\end{table}

Table~\ref{fig:cpu_mem_parallel} describes the resource comparison between two and three simultaneous calls. CPU usage is critical when handling three peer connections or when the network condition forces the congestion mechanism to continuously adapt the bandwidth and encoding. In this test, each call is placed in a different process which should improve the results as the OS will handle them better than in a single thread. When the CPU load gets to its maximum the performance of WebRTC for encoding/decoding and transmission is deprecated, in this kind of topologies having high CPU performance increases the call quality.

\begin{table}[h]
\begin{center}
    \begin{tabular}{c D{,}{\pm}{-1} D{,}{\pm}{-1} D{,}{\pm}{-1} }
   	 \toprule
	\textit{}
	& \multicolumn{1}{c}{\textit{Machine A}}
	& \multicolumn{1}{c}{\textit{Machine B}}
	& \multicolumn{1}{c}{\textit{Overall}}\\
	\midrule
	\textbf{Three calls} & 768.04 ,180.93 & 850.1 ,223.84 & 809.07 ,202.38 \\
	\textbf{Two calls 20 Mbit/s} & 432.56 ,141.32 & 531.13 ,169.82 & 481.85 ,155.56 \\
	\textbf{Two calls 10 Mbit/s} & 178.83 ,60.05 & 141.83 ,42.02 & 160.24 ,51.04 \\
	\textbf{Two calls} & 392.08 ,181.9 & 545.94 ,259.27 & 469.01 ,221.09 \\
	\bottomrule
    \end{tabular}
    \caption[Bandwidth rates for parallel calls in different link conditions]{Bandwidth rates for parallel calls in different link conditions.}
    \label{fig:bw_parallel}
\end{center}
\end{table}

The bandwidth represented in Table~\ref{fig:bw_parallel} is the one used per each machine with the calls together, we will study the worst case. The bandwidth for three calls is to use an average of 800 Kbit/s with different responses in every case, the deviation is approximately $\pm202$ Kbit/s. 

\begin{figure}[h]
  \centering
    \includegraphics[width=1\textwidth]{./figures/sync_three-calls.pdf}
      \caption[Bandwidth representation for all remote streams in a synchronous three peer parallel call for first iteration]{Bandwidth representation for all remote streams in a synchronous three peer parallel call for first iteration.}
	\label{fig:three_parallel}
\end{figure}


Furthermore, it is interesting to see the global rate on the call as the bandwidth averaged is not following a stable value, Figure~\ref{fig:three_parallel} represents the bandwidth during all call for the remote video stream of the three peers. We can see how the rate mechanism tries to use the maximum available rate for the actual video encoding but fails to reach the 2 Mbit/s as multiple calls are running and behaving in the same way, this decision is taken also considering the delay that limits the maximum available rate for the call. Figure represents the delay on the same streams during the call, we can observe those peaks of delay during the same period as the bandwidth rise, the result of this is the sudden drop of bandwidth, this mechanism is triggered multiple times.

\begin{figure}[h]
        \centering
        \begin{subfigure}[b]{0.5\textwidth}
                \centering
                \includegraphics[width=\textwidth]{./figures/delay_three_parallel_1.pdf}
                \caption{Remote stream for call 1}
                \label{fig:three_parallel_1}
        \end{subfigure}%
        ~ %add desired spacing between images, e. g. ~, \quad, \qquad etc.
          %(or a blank line to force the subfigure onto a new line)
        \begin{subfigure}[b]{0.5\textwidth}
                \centering
                \includegraphics[width=\textwidth]{./figures/delay_three_parallel_2.pdf}
                \caption{Remote stream for call 2}
                \label{fig:three_parallel_2}
        \end{subfigure}        
        ~ %add desired spacing between images, e. g. ~, \quad, \qquad etc.
          %(or a blank line to force the subfigure onto a new line)
        \begin{subfigure}[b]{0.5\textwidth}
                \centering
                \includegraphics[width=\textwidth]{./figures/delay_three_parallel_3.pdf}
                \caption{Remote stream for call 3}
                \label{fig:three_parallel_3}
        \end{subfigure}
        \caption[Delay representation for all remote streams in a three peer parallel call]{Delay representation for all remote streams in a three peer parallel call.}
        \label{fig:three_parallel}
\end{figure}

We can compare this scenario with the one in Figure~\ref{fig:1cd81aa8-bw}, the difference relays in the channel condition, in the example of Figure~\ref{fig:1cd81aa8-bw} the channel condition set high restrictions on the path making the rate drop and keep stable as the condition didn't change after that moment. In Figure~\ref{fig:three_parallel}, path condition changes after the drop as it becomes available again as all the three peer calls behaved the same way. 

In general the delay response in all the calls is bad, Figure~\ref{fig:delayThreeCalls} plots the delay distribution of the three simultaneous calls, the slow increase of delay makes the call lag large and variable, probably the user experience for all the three calls will be bad.

\begin{figure}[h]
  \centering
    \includegraphics[width=1\textwidth]{./figures/three_parallel_total_delay_distribution.pdf}
      \caption[Total delay distribution for three parallel calls]{Total delay distribution for three parallel calls.}
	\label{fig:delayThreeCalls}
\end{figure}

The problem relies on the separate treatment of every peer connection, they have no acknowledge of having different similar processes going so the rate change cannot be constrained by the other ongoing calls. Notice that for this test all calls started at the same time, for this purpose we ran a second set of tests starting every call delayed by 15 seconds to check the behavior of the system.

After analyzing the captures of the new asynchronous call we can plot the averaged bandwidth to be approximately 1154 Kbit/s with $\pm250$ Kbit/s of deviation. Overall averaged results are significantly higher than in the previous case but we should have a close look at Figure that represents the bandwidth behavior of the three calls during all the duration for the first iteration we can observe that the average is high but one of the calls rate is very low.

\begin{figure}[h]
  \centering
    \includegraphics[width=1\textwidth]{./figures/async_three-calls.pdf}
      \caption[Bandwidth representation for all remote streams in an asynchronous three peer parallel call for iteration one]{Bandwidth representation for all remote streams in an asynchronous three peer parallel call for iteration one.}
	\label{fig:three_parallel}
\end{figure}

In this case, the first call that started to increase its rate suddenly drops it to approximately 100 Kbit/s and stays there along the call meanwhile the other two parallel calls try to obtain the maximum available bandwidth of the path. This environment is more approximated to the a real scenario as users won't start calls exactly at the same time but they will probably do that randomly, some calls quality will be degraded with barely no quality and others will have sudden drop of bandwidth affecting the interaction with the user.

\begin{figure}[h]
  \centering
    \includegraphics[width=1\textwidth]{./figures/async_total_delay_distribution.pdf}
      \caption[Total delay distribution for three asynchronous parallel calls]{Total delay distribution for three asynchronous parallel calls.}
	\label{fig:delayThreeCallsAsync}
\end{figure}

Figure~\ref{fig:delayThreeCallsAsync} represents the delay distribution for the asynchronous test, similar to the previous example (Figure~\ref{fig:delayThreeCalls}) but with a worst delay response. Delay will affect to the user experience having sudden cuts of communications of milliseconds that will occur randomly, they are not large but they are random and unexpected.

WebRTC should be able to identify parallel connections of the same type and balance the bandwidth usage, this is difficult as the transport level uses already existing RTP technology over UDP, the conclusion is that WebRTC will not be reliable for multiple parallel calls in an average computer.

\subsection{Mesh topology}

A common setup in real-time communications is video conferencing, this way of calling people is widely used in virtual meetings. Until this moment there were multiple available options in the market, with the arrival of WebRTC this feature extends to the web application world, with multiple options and features to be enabled with it. We will try to determine if WebRTC is mature enough to handle multiple peers at the same time and session, the way this is done varies depending on the technology, we will study pure peer-to-peer mesh networks.

The most common option for this kind of environments is to use an MCU to perform the relying of the media through a unique connection by multiplexing the streams in a single one. There are some MCU available in the market for WebRTC but the API is still not evolved enough to allow multiplexing of streams over the same Peer Connection, in the following updates of the API this should be enabled allowing developers to perform real media multiplexing. Some vendors offer MCUs that require extra plugins to be installed, this is due to the impossibility to multiplex multiple media streams over the same Peer Connection, this is required when using Google Hangouts product.

\begin{figure}[h]
  \centering
    \includegraphics[width=0.7\textwidth]{./figures/mesh.pdf}
      \caption[Mesh topology for WebRTC]{Mesh topology for WebRTC.}
	\label{fig:meshTopology}
\end{figure}

Our topology, shown in Figure~\ref{fig:meshTopology}, consist in a mesh network of different virtual machines that connect by using our centralized Dialogue.io application for signaling, media is sent over the peers directly, this will produce a big load in the performance for those clients as the amount of peer connections will be large, each of them obliged to encode and decode media, different from the previous example of parallel calls in this case all the peer connections will be running in the same process making the resource management a key point for the performance of WebRTC. We have increased the amount of CPUs to two in order to get proper results.

Three peers are used for the first test and it will increased by one peer every test, the output for the first test is shown in Table~\ref{fig:mesh_three_peers}. It is important to consider the amount of time required to set up the call, the worst result for the setup time in this scenario has been of 2206 ms to set up the three whole mesh. 

\begin{table}[h]
\begin{center}
    \begin{tabular}{c D{,}{\pm}{-1} D{,}{\pm}{-1} D{,}{\pm}{-1} }
   	 \toprule
	\textit{}
	& \multicolumn{1}{c}{\textit{Machine A}}
	& \multicolumn{1}{c}{\textit{Machine B}}
	& \multicolumn{1}{c}{\textit{Machine C}}\\
	\midrule
	\textbf{CPU (\%)} & 88.5 ,4.77 & 89.49 ,4.46 & 91.65 ,4.23\\
	\textbf{Memory (\%)} & 49.25 ,0.33 & 50.52 ,0.28 & 55.52 ,0.29\\
	\textbf{Bandwidth (Kbit/s)} & 333.38 ,115.13 & 344.48 ,95.43 & 410.77 ,115.97\\
	\bottomrule
    \end{tabular}
    \caption[CPU, memory and bandwidth results for three peer mesh scenario without relay]{CPU, memory and bandwidth results for three peer mesh scenario without relay.}
    \label{fig:mesh_three_peers}
\end{center}
\end{table}

CPU and memory usage is nearly double than in the previous scenario, considering we have doubled the CPU capacity, being this very consuming taking into consideration that is the only application running on the test machine.

%Summary of results
%\include{./chapters/summary}

%Discussion
%\include{./chapters/discussion}

%Real use case (dialogue.io)
%\include{./chapters/dialogue}

%%conclusion chapter
\section{Conclusion}

%% 
%% Leave first page empty
\thispagestyle{empty}

The end.


\addcontentsline{toc}{section}{References}
\bibliographystyle{unsrt}
\bibliography{./chapters/allpapers} 

%%Appendix
\appendix
\section{Setting up fake devices in Google Chrome}
\label{sec:fakeVideo}

%% 
%% Leave first page empty
\thispagestyle{empty}

To address the issue in the video that is transferred from our automated devices we have built a fake input device on the virtual machines that will be fed with a RAW YUV video of different resolutions and quality. This device will be added by using a hacked version of the {\it V4L2Loopback} which derives from the {\it V4L} driver for Linux, the modified version of the {\it V4L2Loopback} builds two extra devices as Chrome is unable to read from the same reading/writing device for security reasons, one of them will be used to fed the video and the other one to read it~\cite{chromiumIssue142568}. 

Differences between standard driver and hacked version:

\begin{itemize}
	\item Need to write a non-null value into the the bus information of the device, this is required as Chrome input needs to be named as a real device. When using Firefox this is not required but works as well.
\end{itemize}

\lstset{language=C}
\begin{lstlisting}
strlcpy(cap->bus_info, "virtual", sizeof(cap->bus_info));
\end{lstlisting}
\begin{itemize}
	\item Our driver will pair devices when they are generated, this will create one read device and one capture device. Everything written into {\it /dev/video0} will be read from {\it /dev/video1}. 
\end{itemize}

\lstset{language=C}
\begin{lstlisting}
cap->capabilities |= V4L2_CAP_VIDEO_OUTPUT | V4L2_CAP_VIDEO_CAPTURE;
\end{lstlisting}

We used the code provided by Patrik H�glund~\cite{chromiumIssue142568} for the {\it V4L2Loopback} hacked version. 

\begin{verbatim}
# make && sudo make install
# sudo modprobe v4l2loopback devices=2
\end{verbatim}

Now we should be able to see both devices in our system, next step is feeding the {\it /dev/video1} with a YUV file. In order to do this we will use the {\it V4l2 File Player}~\cite{v4l2fileplayer}, this player executes on top of {\it Gstreamer} but adds a loop functionality to the file allowing long calls to succeed. Sample videos can be obtained from a Network Systems Lab.\footnote{http://nsl.cs.sfu.ca/wiki/index.php/Video\_Library\_and\_Tools}

\begin{verbatim}
# sudo apt-get install gstreamer0.10-plugins-bad libgstreamer0.10-dev
# make
# v4l2_file_player foreman_cif_short.yuv 352 288 /dev/video1 >& /dev/null
\end{verbatim}

We can now open Google Chrome and check if the fake device is correctly working in any application that uses GetUserMedia API.

\section{Modifying Dummynet for bandwidth requirments}
\label{sec:dummynet}

%% 
%% Leave first page empty
\thispagestyle{empty}

{\it Dummynet} is the tool used to add constraints and simulate network conditions in our tests. 

Besides this, {\it Dummynet} has been natively developed for {\it FreeBSD} platforms and the setup for {\it Linux} environments is sometimes not fully compatible. Our system runs with Ubuntu Server 12.10 with a 3.5.0 kernel version on top of VirtualBox, this system requires to modify some variables and code in order to achieve good test results.

The accuracy of an emulator is given by the level of detail in the model of the system and how closely the hardware and software can reproduce the timing computed by the model~\cite{dummynetRevisited}. Considering that we are using standard Ubuntu images for our virtual machines we will need to modify the internal timer resolution of the kernel in order to get a closer approximation to reality, the default timer in a Linux kernel  2.6.13 and above is 250Hz~\cite{linuxKernelTime}, this value must be changed to 1000Hz in all machines that we intend to run {\it Dummynet}. The change of timing for the kernel requires a full recompile of itself. This change will reduce the timing error from 4ms (default) to 1ms. This change requires the kernel to be recompiled and might take some hours to complete.

Once the kernel timing is done we will need to compile the {\it Dummynet} code, the version we are using in our tests is 20120812, that can be obtained form the {\it Dummynet} project site~\cite{dummynetTool}.

We should try the code first and check if we are able to set queues to our defined pipes, this part is the one that might crash due to system incompatibilities with FreeBSD and old kernel versions of Linux.  If we are unable we should then modify the following code in the {\it ./ipfw/dummynet.c} and {\it ./ipfw/glue.c}.

\lstset{language=C}
\begin{lstlisting}
Index: ipfw/dummynet.c
===================================================================

if (fs->flags & DN_QSIZE_BYTES) {
	size_t len;
	long limit;

	len = sizeof(limit);
	limit = XXX;
	if (sysctlbyname("net.inet.ip.dummynet.pipe_byte_limit", &limit, &len, NULL, 0) == -1)
		limit = 1024*1024;
	if (fs->qsize > limit)
		errx(EX_DATAERR, "queue size must be < \%ldB", limit);
} else {
	size_t len;
	long limit;

	len = sizeof(limit);
	limit = XXX;
	if (sysctlbyname("net.inet.ip.dummynet.pipe_slot_limit", &limit, &len, NULL, 0) == -1)
		limit = 100;
	if (fs->qsize > limit)
		errx(EX_DATAERR, "2 <= queue size <= \%ld", limit);
}
\end{lstlisting}

The problem arises from a the misassumption of {\it sizeof(long) == 4} in 64-bit architectures which is false. By changing those two files we are modifying the system in order to accept higher values than 100 for the queue length.

\lstset{language=C}
\begin{lstlisting}
Index: ipfw/glue.c
===================================================================

char filename[256];	/* full filename */
char *varp;
int ret = 0;		/* return value */
long d;
 
if (name == NULL) /* XXX set errno */
	return -1;


	fprintf(stderr, "\%s fopen error reading filename \%s\n", __FUNCTION__, filename);
	return -1;
}
if (fscanf(fp, "\%ld", &d) != 1) {
	ret = -1;
} else if (*oldlenp == sizeof(int)) {
	int dst = d;
	memcpy(oldp, &dst, *oldlenp);
} else if (*oldlenp == sizeof(long)) {
	memcpy(oldp, &d, *oldlenp);
} else {
	fprintf(stderr, "unknown paramerer len \%d\n",
	(int)*oldlenp);
}
fclose(fp);


	fprintf(stderr, "\%s fopen error writing filename \%s\n", __FUNCTION__, filename);
 	return -1;
 }
if (newlen == sizeof(int)) {
	if (fprintf(fp, "\%d", *(int *)newp) < 1)
		ret = -1;
} else if (newlen == sizeof(long)) {
	if (fprintf(fp, "\%ld", *(long *)newp) < 1)
		ret = -1;
} else {
	fprintf(stderr, "unknown paramerer len \%d\n",
		(int)newlen);
}
fclose(fp);

\end{lstlisting}

When doing this we are making the file compatible with systems that have compatibility problems with the {\it sysctlbyname} function, XXX should be the value of the queue maximum length in slots and Bytes. Slots are defined considering a maximum MTU size of 1500 Bytes.

By default, maximum queue size is set to 100 slots, this amount of slots is not designed for bandwidth demanding tests such as 10Mbit/s or similar. In order to modify this we will need to set a higher value according to the maximum we require. Once this is set we need to recompile {\it Dummynet} form the root directory of the download source code and follow the install instructions in the README file attached to the code.

Even we have allowed {\it Dummynet} to accept more than 100 slots we won't be able to configure them into the pipe even the shell does not complain with error. The next step is to modify the module variables set in the {\it /sys/module/ipfw\_mod/parameters} folder, this folder simulates the {\it sysctl} global variables that we would have running {\it FreeBSD} instead of Linux. 

We need to modify the files {\it pipe\_byte\_limit} and {\it pipe\_slot\_limit} according to the values set in the {\it dummynet.c} previously modified.

Last convenient step is to add ipfw\_mod to the end of {\it /etc/modules} file so {\it Dummynet} module will be loaded even time the system starts. 

We can now set large queues according to our needs.

\section{Scripts for testing WebRTC}
\label{sec:scriptsWebRTC}

%% 
%% Leave first page empty
\thispagestyle{empty}


\lstset{language=bash}
\begin{lstlisting}[caption=Script for testing WebRTC with 15 iterations]
#!/bin/bash
#

#First argument will define the name of the test, second the video to use
#and third may define the IPERF configuration if used

#We also will use this as test example modifying some parameters for the
#other examples such as parallel and mesh


echo "" > 1to1.log
#Exporting variables required for the test
echo "Exporting variables"
PATH="$PATH:/home/lubuntu/MThesis/v4l2_file_player/"
PASSWORD=lubuntu

#Timers for the call duration and break time after the call
REST_TIMEOUT=30
TIMEOUT=300

INIT_TIME=$(date +"%m-%d-%Y_%T")

#Define folders to sabe files
backup_files="/home/lubuntu/MThesis/ConMon/rtp/rtp_*"
mkdir results/$INIT_TIME"_"$1
dest_folder="/home/lubuntu/results/"$INIT_TIME"_"$1
echo "Starting $INIT_TIME"
counter=0

#Loop the test 15 times to avoid call failures
while [ $counter -le 14 ]
do
	actual_time=$(date +"%m-%d-%Y_%T")
	echo "Iteration - $counter"
 	#Clean all ongoing processes from previous iterations
 	echo "Cleaning processes"
 	echo $PASSWORD | sudo -S killall conmon >> 1to1.log 2>&1
 	killall v4l2_file_player >> 1to1.log 2>&1
 	killall chrome >> 1to1.log 2>&1
	sleep $REST_TIMEOUT
 
 	#Set virtual device for Webcam
 	echo "Setting dummy devices"
 	echo $PASSWORD | sudo -S modprobe v4l2loopback devices=2 >> 1to1.log 2>&1

 	cd MThesis/ConMon
 	#Start ConMon and configure 192.168.1.106 which is the turn relay for the media
 	echo $PASSWORD | sudo -S ./conmon eth3 "udp and host 192.168.1.106" --turn >> 1to1.log 2>&1 &
 	cd ../..

 	#Load fake video into virtual device
 	echo "Loading video"
 	v4l2_file_player /home/lubuntu/MThesis/v4l2_file_player/$2 352 288 /dev/video1 >> 1to1.log 2>&1 &
 	#If third argument available then we run the IPERF
 	if [ $# -eq 3 ]
 	then 
        		iperf -c 192.168.1.106 -t 300 -i 5 -b $3 >> 1to1.log 2>&1 &
 	fi
 
 	#Load browser pointing the test site with the n= parameter that will define the StatsAPI filename
 	#We need to ignore the certificate errors to load the page with an untrusted certificate
 	DISPLAY=:0 google-chrome --ignore-certificate-errors https://192.168.1.100:8088/?n=$1"_"$counter >> /dev/null  2>&1 &

 	#Script for capturing CPU and Memory usage for every test
 	./memCPU.sh $dest_folder $counter >> 1to1.log 2>&1 &
 	memCPUPID=$!

 	sleep $TIMEOUT
 	echo $PASSWORD | sudo -S killall conmon >> 1to1.log 2>&1
 	kill $memCPUPID
 	dir_file=$1"_"$counter
 	mkdir $dest_folder/$dir_file
 	mv $backup_files $dest_folder/$dir_file
 	(( counter++ ))
done

sleep 30
echo "Finishing test..."
echo $PASSWORD | sudo -S killall conmon >> 1to1.log 2>&1
killall v4l2_file_player >> 1to1.log 2>&1
killall chrome >> 1to1.log 2>&1
\end{lstlisting}

\lstset{language=bash}
\begin{lstlisting}[caption=Measue and store CPU and Memory usage]
#!/bin/bash 

#Script used to measure periodically the status of the CPU and memory
PREV_TOTAL=0 
PREV_IDLE=0 

#Runs until the script is killed by another process
while true; 
do 
  	CPU=(`cat /proc/stat | grep '^cpu '`) # Get the total CPU statistics. 
  	unset CPU[0]                          # Discard the "cpu" prefix. 
  	IDLE=${CPU[4]}                        # Get the idle CPU time. 
  	timeStamp=$(date +%s)
  	# Calculate the total CPU time. 
  	TOTAL=0 
  	for VALUE in "${CPU[@]}"; do 
    		let "TOTAL=$TOTAL+$VALUE" 
  	done 

  	# Calculate the CPU usage since we last checked. 
  	let "DIFF_IDLE=$IDLE-$PREV_IDLE" 
  	let "DIFF_TOTAL=$TOTAL-$PREV_TOTAL" 
  	let "DIFF_USAGE=(1000*($DIFF_TOTAL-$DIFF_IDLE)/$DIFF_TOTAL+5)/10" 

	# Remember the total and idle CPU times for the next check. 
 	PREV_TOTAL="$TOTAL" 
 	 PREV_IDLE="$IDLE" 

  	#Save the amount of used memory in Mb
  	total=$(free |grep Mem | awk '$3 ~ /[0-9.]+/ { print $2"" }')
  	used=$(free |grep Mem | awk '$3 ~ /[0-9.]+/ { print $3"" }')
  	free=$(free |grep Mem | awk '$3 ~ /[0-9.]+/ { print $4"" }')
  	
	#Calculate the percentage
  	usedmem=`expr $used \* 100 / $total`

  	#Export all the data to the defined iteration in argument 2 and folder 1
  	echo  $timeStamp"     "$DIFF_USAGE"  "$usedmem"      "$total"        "$used" "$free >> $1/log_performance_$2.txt

  	# Wait before checking again one second
  	sleep 1 
done
\end{lstlisting}


\end{document}


