\section{Performance Metrics for WebRTC}


%% 
%% Leave first page empty
\thispagestyle{empty}

This section will define the way we measure the performance in WebRTC environments, this real-time media environment will require an specific approach and some metrics to define how the protocol behaves in different topologies and scenarios. 

Different issues might affect directly how the WebRTC media performs, these range from the hardware of the clients to the state of the link. In the following chapters we will describe some of them that will be used in our study cases.

\subsection{Losses}

Loss rate indicates packet losses during the transmission or processing. Usually packet losses affect directly the performance of a call and can indicate how the link is behaving between the different peers, in our case, packet loss will be a direct indicator of the quality of the ongoing WebRTC transmission. However, the packet loss indicate that some packets are not arriving, another strong indicator that goes attached is delay as packets will arrive later prior to getting lost in the link. This indicator will show up when the link is carrying big congestion of failures. 

The second option is called congestion error meanwhile the loss of packets by buffered queues or other issues is called bit-error, over use of a link may led to losses due to congestion. In WebRTC we are using RTCP protocol for control and reporting of the ongoing stream~\cite{rtcpIETF}.

Some delayed packets should also be considered as losses as they won't be useful anymore for the ongoing connection, those packets won't show up in the stats as losses. In WebRTC loss rate will affect directly to the ongoing transmission as the delay range that we can tolerate is very low before the quality of the call deteriorates, some data-driven WebRTC connections will tolerate some more delay. In general case Loss Rate will be considered as a main point for recalculating the path by using faster routes. This indicator is manly attached to link quality.

Losses will be calculated in a certain period of time so we will be able to see how much loss rate we have in a certain range of time.

\begin{equation}
	\frac{PKT_{loss}(T) - PKT_{loss}(T-1)}{PKT_{received}(T) - PKT_{received}(T-1) + PKT_{loss}(T) - PKT_{loss}(T-1)}
	\label{eq:PKTloss}
\end{equation}

Equation~\ref{eq:PKTloss} calculates the estimated packet loss we might have on the link. This operation will be done every period, we will determine this period when building the testing environment.

\subsection{Round-Trip Time (RTT) and One-way delay (OWD)}

The delay in a link can be measured form different perspectives, one-way delay indicates the time it takes for a packet to move from one peer to the other peer, this time includes different delays that are given in the link. This one-way delay is calculated form the time taken to process it in both sides (building and decoding), the lower layer delay in the client (interface and intra-layering delay), queuing delay (from the multiple buffers in the path) and propagation delay (speed of light). The sum of all those delays compose the total one-way delay.

Considering the structure of WebRTC, one of the most important delays that we will have to consider and study is the processing delay as our applications will rely in a multiple layer structure, running over the browser will affect the performance compared to other technologies that run directly over the OS. Delays in our case will be symmetric as we will be sending and receiving media, the delay will be important in order to reproduce the streams in the best quality possible and avoid decoding artifacts in the media. 

OWD and RTT measurements are included in standard RTCP specification, in order to calculate this timestamp from sender and receiver is needed in the reports. Sender Report (ST) timestamp is saved in the sender meanwhile the receiver informs the same timestamp in the Receiver Report (RR) that goes back to the origin. By that, we are able to calculate the RTT using the following Equation~\ref{eq:rtt}.

\begin{equation}
	RTT = TS_{RR} - TS_{SR} -T_{Delay}\\
	
	TS_{RR} \text{: Local timestamp at reception of last Receiver Report}\\
	
	TS_{SR}�\text{: Last Sender Report timestamp}\\
	
	T_{Delay} \text{: Receiver time period between SR reception and RR sending in the sender}\\
	\label{eq:rtt}
\end{equation}

Calculating OWD requires both machines clock to be accurately synchronized, we might try to assure this but usually OWD delay is defined as $\frac{RTT}{2}$. 

RTT will be an early indicator of congestion in a WebRTC connection, this RTT must be monitored and most important, the adequate RTT have to be defined for every connection as the clients won't be aware of the appropriate amount for good performance.

\subsection{Throughput}

Throughput will be a key metric for testing the performance of WebRTC environments, this value will show how much capacity of the link is taking each PC and stream. It is complex though as there is still no QoS implemented in WebRTC. The throughput metric is going to provide bandwidth for video/audio in each direction, we can then use this value to provide some quality metric averaging all the previous mentioned measures in order to monitor the overall quality of the call. A sudden drop of the throughput will mean that the bandwidth available for that PC has been drastically reduced, this will lead to artifacts, or in the word case, loose of communication between peers. In this specific situation ICE candidates will try to be renegotiated in order to obtain a different solution for the connection and reestablish the media with the best throughput possible.

Furthermore, throughput can be divided into sending rate ($BR_S$), receiver rate ($BR_R$) and goodput ($GP$). From the technical point of view sender rate is defined as the amount of packets that are injected into the network by the sender, receiver rate is the speed at which packets arrive at the receiver and goodput is calculated by discarding all the lost packets on the path, only packets that have been received are counted making goodput a good metric for our purposes. Usually, those metrics are calculated by extracting the information from the RTCP packets, in our case, we will rely on the Stats API that uses the WebRTC API to obtain the amount of bytes and measurements to manually calculate those by using JavaScript. Taking into account the last amount of bytes received, the actual amount and the time elapsed we will be able to calculate and accurate value for the goodput or throughput.

\subsubsection{Audio streams}

When using real time media environments for bidirectional communication the user experience is a key indicator of success. One of the factors that have to be considered is the Noise Reduction (NR) and Acoustic Echo Canceler (AEC). Those mechanisms allow the call to be smooth and avoid extra noises and echoes from the speaker voice to be transmitted, in WebRTC will provide a strange behavior when measuring the throughput, when the is no speech the bytes transferred will be approximately zero, being the throughput negligible. This helps to reduce the bandwidth usage and provides a more comfortable conversation when having a call.

\subsection{Other metrics}

Besides the metrics explained in the previous sections we are keeping other important values that affect WebRTC.

CPU/RAM usages are logged in order to determine how an average system performs when running the different scenarios as some of them will be more demanding than others, this will give an approximate approach to the required resources needed.

Also call setup time and frequency of call drops will be saved, it might be important to determine an approximate call setup time since the start of the negotiation until the media arrives. By doing this, we are measuring a parameter that directly affects the user experience in WebRTC. From the other side, call drops will be counted to see the call success ratio in every scenario.

We will also calculate and pay attention to the delay variation, this is important as it affects how the user interacts with the other peer during a call. Having high delay variations led to an uncomfortable call and distortion. We will measure this variation and the amount of delay that different topologies produce.

\subsection{Summary of metrics}

Performance metrics for WebRTC will help us to determine the behavior of the link just by using information provided by the RTCP packets and the pure API, the goal of all this metrics is to provide better mechanisms to properly adapt the rate and response to the condition of the link. Rate adaptation will be a key mechanism to provide good response in WebRTC and we will closely study how those perform in different environments, some such as RTT, OWD and setup call time will affect the user experience more than the pure video quality of the call and should also be taken in consideration. Based on the results we get for those indicators we will determine wether the congestion mechanisms are worked as defined or should be improved in the actual version of WebRTC.