\section{Conclusion}

%% 
%% Leave first page empty
\thispagestyle{empty}

In this thesis we have analyzed and evaluated WebRTC protocol for real-time web applications in different environments. We also compared this protocol with already existing real-time communications alternatives and described the usage of the actual implementation of WebRTC. Furthermore, we chose a set of key indicators that are important when measuring the performance of a RTC protocol. Those indicators are used when evaluating the congestion control mechanisms of WebRTC. We also described the possible real-time topologies used to test the performance of WebRTC, some of them are still not possible to implement due to API constraints. To evaluate the protocol we also built a specific setup for this thesis, the environment is used during the development of all the thesis. Finally, we executed the tests after describing the proposed congestion control mechanisms and the ones already implemented in WebRTC.

After the tests, we can conclude that WebRTC is a solid protocol for real-time web applications that performs correctly in constrained environments with low latencies, up to 200ms, but cannot hold greater values. This condition may affect mobile applications relying on WebRTC. The congestion control mechanism implemented in WebRTC, Receiver-side Real-time Congestion Control (RRTCC), copes correctly with packet losses protecting the packets with different mechanisms such as Forward Error Correction (FEC). To improve the performance of WebRTC, different congestion control mechanisms that react to other indicators could be implemented in the internals of the browsers. Given the results obtained in this thesis, we can conclude that RRTCC is still unfinished and requires to evolve further to fit all the requirement of different real time scenarios.

Future researches could be done in order to determine the usability limits of WebRTC. Future works include a deeper experience in mobile environments with the two possible implementations: native mobile application and mobile web application. Both are crucial for the expansion of WebRTC into mobile platforms. Besides this, the analysis of WebRTC session on mobile moving nodes is also important to check the response of the call quality.

WebRTC APIs also need enhancements in order to test all possible environments. One of the most interesting areas to investigate on is the development of a transcoding MCU in WebRTC. During this thesis we have used a packet relying MCU environment for the tests, the transcoding MCU multiplexes and mixes the media streams into a unique channel to improve performance over the path. This type of environment is commonly used for multiparty calls. A better understanding of the Stats API implemented in WebRTC is also needed to better adequate the constraints of the media acquisition and the PeerConnection API. Those metrics are provided by the browser internals and can provide valuable feedback for the application developer that can be used to improve the quality of the session.

Lastly, a follow-up of the cross-compatibility of the browser congestion control mechanisms should be studied in order to provide full interoperability between vendors and devices. At the development of this thesis, this compatibility is not achieved due to the lack of congestion control mechanisms in some browser providers.