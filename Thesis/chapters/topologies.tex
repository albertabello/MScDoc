\section{Topologies}
\label{sec:topologies}


%% 
%% Leave first page empty
\thispagestyle{empty}

We will talk about the different topologies that will be studied in this document that use WebRTC.

\subsection{Point-to-Point}

Point-to-point environments are widely used as a private calling system between two individuals, when related to WebRTC those use cases can be extended to people in the same domain so it is not required to be as private as other technologies. 

Support systems are being considered as a valuable scenario for WebRTC point-to-point calls, this method will be useful as WebRTC does not need any plugin or account setup if we want an easy way to provide support to our customers in an HTML5 enabled site.

Other similar uses could be communication between doctor and patient in a medical application that intends to be cross-platform compatible in an ongoing standard. Communication in other scenarios such as citizens and authorities could also rely in a WebRTC application.
 
\subsubsection{Challenges}

In this scenario most challenges will come from the networking aspect. We could consider the enterprise use case, we will have ongoing calls between workers in the same enterprise network, this network could be restrictive to the use of non-trusted STUN servers or it might drop the UDP packets. Having restrictive NATs or Firewalls will directly affect the possibilities of having the WebRTC call correctly established, in this situation the possibilities of success highly rely on the network used.

When having point-to-point calls the problem that could arise from the local performance on the machine is not that critical considering that only one peer connection will be handled carrying two to three streams. We will focus on the user experience when having high restrictions in the NAT/FW.

\subsection{One-to-Many}

One-to-many scenarios are widely known as a type of multicast, one source sends the media to the different clients that connect to the origin. When having this topology the common uses rely on video and audio streaming to multiple clients, TV channels and streaming conferences are popular.

For example, we could have a major sport even being retransmitted to the viewers by using one-to-many. Other solutions could cover the use of WebRTC to have a CEO talking to the employees by using an HMTL5 web application. Music bands also could take advantage of this scenario by being able to transmit his show to the audience.

\subsubsection{Challenges}

In this scenario we will have a video, audio and data streaming connection from one source to multiple devices. This will cause a huge load on the source when having multiple PeerConnections running, local performance will be a constraint. Considering other scenarios, in this case, latency on the network is not as important as the rest due to the one-way communication only, no video and audio is needed to be received on the source.

We will need to study ICE and STUN mechanisms and how they perform in this scenario but the problems will arise from the source capacity on the hardware and link. Bandwidth demand on the source will be high and may affect the communication. On the other hand, having the audio delayed a couple of seconds is not going to affect the user experience in the call.

From the client perspective, the PeerConnection stablished will be easy to handle as no RTP streams will be sent back to the source, except the RTCP messages.

\subsection{Many-to-Many}

Many-to-many topologies are used for conferencing environment, this topology is used in systems such as Skype or VoIP. Conferences are used in enterprises for long-distance communication between employees and working groups, by this, the need of having those calls working smoothly for all participants is very important.

Due to the compatibility with HTML5 and the DataChannel spec that is shipped with WebRTC many-to-many environment could be also used for data transmission between different peers in a torrent scenario. Combining it with WebGL it could even be used for gaming experiences or file sharing.

\subsubsection{MCU}

MCU usage will surely be an option when designing WebRTC infrastructures, the ability to multiplex different streams into the same channel will directly affect on how the client performs when reproducing the video reducing the amount of used resources. 

\subsubsection{Challenges}

When running multiple peer connections browsers on the client side will be obliged to handle multiple incoming streams and keep up all the connections stablished, this will be a problem regarding to the resources on the client side. 

So, alongside with other problems mentioned in previous topologies, resources will directly affect on how this scenario performs in different users. 

On the other side, even considering that NAT and Firewall connectivity issues are solved, we should be careful when guaranteeing all peer connections to be correctly stablished. TURN/STUN failures might directly affect to the success ratio in this topology.

