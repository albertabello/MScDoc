\section{Setting up fake devices in Google Chrome}
\label{sec:fakeVideo}

%% 
%% Leave first page empty
\thispagestyle{empty}

To address the issue in the video that is transferred from our automated devices we have built a fake input device on the virtual machines that will be fed with a RAW YUV video of different resolutions and quality. This device will be added by using a hacked version of the {\it V4L2Loopback} which derives from the {\it V4L} driver for Linux, the modified version of the {\it V4L2Loopback} builds two extra devices as Chrome is unable to read from the same reading/writing device for security reasons, one of them will be used to fed the video and the other one to read it~\cite{chromiumIssue142568}. 

Differences between standard driver and hacked version:

\begin{itemize}
	\item Need to write a non-null value into the the bus information of the device, this is required as Chrome input needs to be named as a real device. When using Firefox this is not required but works as well.
\end{itemize}
\begin{lstlisting}
strlcpy(cap->bus_info, "virtual", sizeof(cap->bus_info));
\end{lstlisting}
\begin{itemize}
	\item Our driver will pair devices when they are generated, this will create one read device and one capture device. Everything written into {\it /dev/video0} will be read from {\it /dev/video1}. 
\end{itemize}
\begin{lstlisting}
cap->capabilities |= V4L2_CAP_VIDEO_OUTPUT | V4L2_CAP_VIDEO_CAPTURE;
\end{lstlisting}

We used the code provided by Patrik H�glund~\cite{chromiumIssue142568} for the {\it V4L2Loopback} hacked version. 

\begin{verbatim}
# make && sudo make install
# sudo modprobe v4l2loopback devices=2
\end{verbatim}

Now we should be able to see both devices in our system, next step is feeding the {\it /dev/video1} with a YUV file. In order to do this we will use the {\it V4l2 File Player}~\cite{v4l2fileplayer}, this player executes on top of {\it Gstreamer} but adds a loop functionality to the file allowing long calls to succeed. Sample videos can be obtained from a Network Systems Lab.\footnote{http://nsl.cs.sfu.ca/wiki/index.php/Video\_Library\_and\_Tools}

\begin{verbatim}
# sudo apt-get install gstreamer0.10-plugins-bad libgstreamer0.10-dev
# make
# v4l2_file_player foreman_cif_short.yuv 352 288 /dev/video1 >& /dev/null
\end{verbatim}

We can now open Google Chrome and check if the fake device is correctly working in any application that uses GetUserMedia API.