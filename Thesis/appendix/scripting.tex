\section{Scripts for testing WebRTC}
\label{sec:scriptsWebRTC}

%% 
%% Leave first page empty
\thispagestyle{empty}


\lstset{language=bash}
\begin{lstlisting}[caption=Script for testing WebRTC with 15 iterations]
#!/bin/bash
#

#First argument will define the name of the test, second the video to use
#and third may define the IPERF configuration if used

#We also will use this as test example modifying some parameters for the
#other examples such as parallel and mesh


echo "" > 1to1.log
#Exporting variables required for the test
echo "Exporting variables"
PATH="$PATH:/home/lubuntu/MThesis/v4l2_file_player/"
PASSWORD=lubuntu

#Timers for the call duration and break time after the call
REST_TIMEOUT=30
TIMEOUT=300

INIT_TIME=$(date +"%m-%d-%Y_%T")

#Define folders to sabe files
backup_files="/home/lubuntu/MThesis/ConMon/rtp/rtp_*"
mkdir results/$INIT_TIME"_"$1
dest_folder="/home/lubuntu/results/"$INIT_TIME"_"$1
echo "Starting $INIT_TIME"
counter=0

#Loop the test 15 times to avoid call failures
while [ $counter -le 14 ]
do
	actual_time=$(date +"%m-%d-%Y_%T")
	echo "Iteration - $counter"
 	#Clean all ongoing processes from previous iterations
 	echo "Cleaning processes"
 	echo $PASSWORD | sudo -S killall conmon >> 1to1.log 2>&1
 	killall v4l2_file_player >> 1to1.log 2>&1
 	killall chrome >> 1to1.log 2>&1
	sleep $REST_TIMEOUT
 
 	#Set virtual device for Webcam
 	echo "Setting dummy devices"
 	echo $PASSWORD | sudo -S modprobe v4l2loopback devices=2 >> 1to1.log 2>&1

 	cd MThesis/ConMon
 	#Start ConMon and configure 192.168.1.106 which is the turn relay for the media
 	echo $PASSWORD | sudo -S ./conmon eth3 "udp and host 192.168.1.106" --turn >> 1to1.log 2>&1 &
 	cd ../..

 	#Load fake video into virtual device
 	echo "Loading video"
 	v4l2_file_player /home/lubuntu/MThesis/v4l2_file_player/$2 352 288 /dev/video1 >> 1to1.log 2>&1 &
 	#If third argument available then we run the IPERF
 	if [ $# -eq 3 ]
 	then 
        		iperf -c 192.168.1.106 -t 300 -i 5 -b $3 >> 1to1.log 2>&1 &
 	fi
 
 	#Load browser pointing the test site with the n= parameter that will define the StatsAPI filename
 	#We need to ignore the certificate errors to load the page with an untrusted certificate
 	DISPLAY=:0 google-chrome --ignore-certificate-errors https://192.168.1.100:8088/?n=$1"_"$counter >> /dev/null  2>&1 &

 	#Script for capturing CPU and Memory usage for every test
 	./memCPU.sh $dest_folder $counter >> 1to1.log 2>&1 &
 	memCPUPID=$!

 	sleep $TIMEOUT
 	echo $PASSWORD | sudo -S killall conmon >> 1to1.log 2>&1
 	kill $memCPUPID
 	dir_file=$1"_"$counter
 	mkdir $dest_folder/$dir_file
 	mv $backup_files $dest_folder/$dir_file
 	(( counter++ ))
done

sleep 30
echo "Finishing test..."
echo $PASSWORD | sudo -S killall conmon >> 1to1.log 2>&1
killall v4l2_file_player >> 1to1.log 2>&1
killall chrome >> 1to1.log 2>&1
\end{lstlisting}

\lstset{language=bash}
\begin{lstlisting}[caption=Measue and store CPU and Memory usage]
#!/bin/bash 

#Script used to measure periodically the status of the CPU and memory
PREV_TOTAL=0 
PREV_IDLE=0 

#Runs until the script is killed by another process
while true; 
do 
  	CPU=(`cat /proc/stat | grep '^cpu '`) # Get the total CPU statistics. 
  	unset CPU[0]                          # Discard the "cpu" prefix. 
  	IDLE=${CPU[4]}                        # Get the idle CPU time. 
  	timeStamp=$(date +%s)
  	# Calculate the total CPU time. 
  	TOTAL=0 
  	for VALUE in "${CPU[@]}"; do 
    		let "TOTAL=$TOTAL+$VALUE" 
  	done 

  	# Calculate the CPU usage since we last checked. 
  	let "DIFF_IDLE=$IDLE-$PREV_IDLE" 
  	let "DIFF_TOTAL=$TOTAL-$PREV_TOTAL" 
  	let "DIFF_USAGE=(1000*($DIFF_TOTAL-$DIFF_IDLE)/$DIFF_TOTAL+5)/10" 

	# Remember the total and idle CPU times for the next check. 
 	PREV_TOTAL="$TOTAL" 
 	 PREV_IDLE="$IDLE" 

  	#Save the amount of used memory in Mb
  	total=$(free |grep Mem | awk '$3 ~ /[0-9.]+/ { print $2"" }')
  	used=$(free |grep Mem | awk '$3 ~ /[0-9.]+/ { print $3"" }')
  	free=$(free |grep Mem | awk '$3 ~ /[0-9.]+/ { print $4"" }')
  	
	#Calculate the percentage
  	usedmem=`expr $used \* 100 / $total`

  	#Export all the data to the defined iteration in argument 2 and folder 1
  	echo  $timeStamp"     "$DIFF_USAGE"  "$usedmem"      "$total"        "$used" "$free >> $1/log_performance_$2.txt

  	# Wait before checking again one second
  	sleep 1 
done
\end{lstlisting}
